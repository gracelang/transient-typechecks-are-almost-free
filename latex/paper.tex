\documentclass[sigplan, review]{acmart}

\usepackage{booktabs} % For formal tables
\usepackage{outlines} % 

% Copyright
%\setcopyright{none}
%\setcopyright{acmcopyright}
%\setcopyright{acmlicensed}
\setcopyright{rightsretained}
%\setcopyright{usgov}
%\setcopyright{usgovmixed}
%\setcopyright{cagov}
%\setcopyright{cagovmixed}


% DOI
\acmDOI{10.475/123_4}

% ISBN
\acmISBN{123-4567-24-567/08/06}

%Conference
\acmConference[WOODSTOCK'97]{ACM Woodstock conference}{July 1997}{El
  Paso, Texas USA}
\acmYear{1997}
\copyrightyear{2016}

\acmPrice{15.00}

%\acmBadgeL[http://ctuning.org/ae/ppopp2016.html]{ae-logo}
%\acmBadgeR[http://ctuning.org/ae/ppopp2016.html]{ae-logo}


% Extras
\newcommand{\unsure}[1]{\textcolor{red}{#1}}
\newcommand{\note}[1]{\textcolor{gray}{#1}}
% \newcommand{\rrnote}[1]{\textcolor{blue}{#1}}  % replaced by much nicer macro

\def\SOMns{SOM{\sc ns}\xspace}

\newcommand{\ie}{i.e.\xspace}
\newcommand{\eg}{e.g.\xspace}

\newcommand{\code}[1]{\texttt{#1}}
% \newcommand{\code}[1]{\lstinline!#1!}

\usepackage{letltxmacro}
\LetLtxMacro{\OrgCitep}{\citep}
\renewcommand{\citep}[1]{\,\OrgCitep{#1}}

% define \citeurl
%  #1 link text (for HTML version)
%  #2 title
%  #3 author
%  [#4] date of retrival
%  #5  url
% Example:
%   \citeurl{linktext}{Welcome to the Jungle}{Herb Sutter}{27 June 2012}{http://herbsutter.com/welcome-to-the-jungle/}
% Result:
%   \footnote{\emph{Welcome to the Jungle}, Herb Sutter, access date: 27 June 2012
%     \url{http://herbsutter.com/welcome-to-the-jungle/}}
\usepackage{ifthen}
\newcommand{\citeurl}[5]{%
#1\footnote{\emph{#2}%
          \ifthenelse{\equal{#3}{}}%
                     {}%          then i.e. empty #3
                     {, #3}% else i.e. non-empty #3
          \ifthenelse{\equal{#4}{}}%
                     {}%          then i.e. empty #4
                     {, access date: #4}% else i.e. non-empty #4
, \url{#5}}}

\newif\ifincludeoutlines
\newif\ifincludecontent
\newif\ifincludeintro
\newif\ifincludebackground
\newif\ifincludemethod
\newif\ifincluderesults
\newif\ifincludeconclusion

\includeoutlinestrue
\includecontenttrue
\includeintrotrue
\includebackgroundtrue
\includemethodtrue
\includeresultstrue
\includeconclusiontrue


\usepackage{collab}  % [hideall]
\collabAuthor{rr}{green!60!black}{Richard}
\collabAuthor{sm}{red}{Stefan}
\collabAuthor{kjx}{orange}{James}

\usepackage[nameinlink]{cleveref}

\begin{document}
\title{SIG Proceedings Paper in LaTeX Format}
\titlenote{Produces the permission block, and
  copyright information}
\subtitle{Extended Abstract}
\subtitlenote{The full version of the author's guide is available as
  \texttt{acmart.pdf} document}

\author{Richard}
\authornote{The one who did the work.}
\orcid{}
\affiliation{%
    \department{}
    \institution{}
    \city{}
    \postcode{}
    \country{}
}
\email{}


\author{Stefan Marr}
\authornote{SM: Can go 2nd or 3rd, doesn't matter for me.}
\orcid{0000-0001-9059-5180}
\affiliation{%
    \department{School of Computing}
    \institution{University of Kent}
    % \city{Canterbury}
    % \postcode{CT2 7NZ}
    \country{United Kingdom}
}
\email{s.marr@kent.ac.uk}

\author{Michael}
\orcid{}
\affiliation{%
    \department{}
    \institution{}
    \city{}
    \postcode{}
    \country{}
}
\email{}

\author{James}
\authornote{SM: I assume James wants to go last (supervisor position).}
\orcid{}
\affiliation{%
    \department{}
    \institution{}
    \city{}
    \postcode{}
    \country{}
}
\email{}

% The default list of authors is too long for headers.
\renewcommand{\shortauthors}{R. Roberts et al.}


\begin{abstract}

%kjx I kent beck-ified it again
%
  Languages with optional explcit dynamic type checking, like
  Typescript, Dart, Groovy, and Grace are increasingly popular in
  practical development and programming education.  Unfortunately,
  current implementations of these languages perform worse than either
  purely statically or purely dynamically typed languages.  We show
  how virtual machines can use common optimisations to remove
  redundant dynamic type checks, by adding dynamic type
  checks to Moth, a Truffle-based interpreter for  Grace.
% or "for thr Grace programming language."
  Moth runs programs with dynamic type checks
  roughly as fast as programs without checks, so developers do not
  need to disable checks in production code, and
  educators can teach types without also teaching that types slow
  programs down.

%% Removing \kjx{redundant} type tests makes programs in these
%% languages run faster. 
% We add shallow-structural type checking
% to an existing interpreter Grace, 
%that enables developers to check 
%that executions of their program are well-typed. % WTF else would you
%need 'em
%
% Grace is designed primarily for education and, consequently,
% the performance profile of its execution is important
% to enable both teachers and students to explore different concepts.
%
% Following from Grace's design goals,
% our focus is to provide flexible checking
% with minimal overhead to execution.
% We achieve flexibility through our shallow-structural design and,
% for our implementation, 
% we achieve minimal overhead by
% taking advantage of common JIT VM techniques 
% that can remove much of the overhead of dynamic type checking;
% allowing programs with explicit dynamic checks 
% to run at roughly the same speed as programs without those checks.
% Overall, adopting common JIT techniques can make these languages
% more effective in practice.

\end{abstract}


%
% The code below should be generated by the tool at
% http://dl.acm.org/ccs.cfm
% Please copy and paste the code instead of the example below.
%
\begin{CCSXML}
<ccs2012>
 <concept>
  <concept_id>10010520.10010553.10010562</concept_id>
  <concept_desc>Computer systems organization~Embedded systems</concept_desc>
  <concept_significance>500</concept_significance>
 </concept>
 <concept>
  <concept_id>10010520.10010575.10010755</concept_id>
  <concept_desc>Computer systems organization~Redundancy</concept_desc>
  <concept_significance>300</concept_significance>
 </concept>
 <concept>
  <concept_id>10010520.10010553.10010554</concept_id>
  <concept_desc>Computer systems organization~Robotics</concept_desc>
  <concept_significance>100</concept_significance>
 </concept>
 <concept>
  <concept_id>10003033.10003083.10003095</concept_id>
  <concept_desc>Networks~Network reliability</concept_desc>
  <concept_significance>100</concept_significance>
 </concept>
</ccs2012>
\end{CCSXML}

\ccsdesc[500]{Computer systems organization~Embedded systems}
\ccsdesc[300]{Computer systems organization~Redundancy}
\ccsdesc{Computer systems organization~Robotics}
\ccsdesc[100]{Networks~Network reliability}


\keywords{ACM proceedings, \LaTeX, text tagging}

% \begin{teaserfigure}
%   \includegraphics[width=\textwidth]{sampleteaser}
%   \caption{This is a teaser}
%   \label{fig:teaser}
% \end{teaserfigure}

\maketitle

\rrtodo{REMOVE: \cite{Richards2017}}

%!TEX root = ../latex/paper.tex

\section{Introduction}
\label{sec:introduction}

% Despite optional typing becoming increasingly popular in the industry, the systems that offer support for optional typing are not yet fully \emph{alive}. The languages are of course well defined and supported but, in particular, the advantages of the optional typing are out-weighted by their performance costs.
%
% Recently, researchers have explored the potential for type checking to be performed together with execution; rather than as an a stand-alone system that occurs separately at run time. For example, Richards et al. extend the Higgs VM to perform TypeScript's structural checks based on information gathered during method execution.
%
% Tying the type checking system into existing VM behavior can offer faster performance, while also being simple to implement. In particular a compiler can fold the type checking system into the VM's invocation logic and, once compiled, both the number of type checks and their complexity has been reduced. Furthermore, the VM usually has cheap reflective access and therefore the logic of the checking system is simpler to implement than we done on the level of the guest language.

% Context
% - dynamic languges are important
% - much work on performance
Dynamic languages are increasingly prominent in the software industry.
Building on the pioneering work of Self\citep{Self}, 
much work in academia and industry
has gone into making them more efficient\citep{Bolz2013,Bolz:2013:IMT,Wurthinger:2017:PPE,Daloze2016,Clifford:2015:MM,Degenbaev:2016:ITG}.
Just-in-time compilers have taken JavaScript, for example, from a
na{\"\i}vely interpreted language barely suitable for browser scripting, 
to a highly efficient ecosystem that is sweeping across industry
and academia\citep{githut2018}.

% Types
% - useful for programmer productivity on large systems
% - document a program's structures
% - optional + gradual
% - allow partial typing
% - are either removed, or have run-time overhead

With these performance gains,
dynamic languages are used to build larger and larger systems,
which leads to typing approaches being adopted
to support programmer productivity and document a program's structures.
Two important approaches are optional\citep{GiladPluggable2004}
and gradual typing\citep{Siek2006,Siek2015}.
These are applied to dynamic languages to reap the benefits of typing, but
unfortunately also have limitations.
With optional or pluggable approaches such as 
%\citeurl{TypeScript,}{TypeScript}{Microsoft}{27 June
%2018}{https://www.typescriptlang.org/}
TypeScript\citep{typeScriptECOOP,GiladPluggable2004}
%
types are erased before the execution,
limiting the benefit of types to the statically typed parts of programs.
In contrast, gradual type systems retain types until run time,
performing the checks dynamically, and
can give detailed information about type violations via blame 
tracking\citep{Siek2015,blame2009}.
Unfortunately, these gradual systems currently impose significant
run-time overheads
\citep{Takikawa2016,Vitousek2017,Muehlboeck2017,Bauman2017,Richards2017,Stulova2016,Greenman2018}.

% our context
% - dynamic language
% - a language for teaching
% - types, execution semantics, consistency
% - no impact on performance

%With the \citeurl{Grace language,}{The Grace Programming
%  Language}{}{}{http://gracelang.org/}
We are working on Grace\citep{graceOnward12}, a dynamic language in
the tradition of Smalltalk\citep{bluebook}, Self\citep{Self}, and
JavaScript that is meant for use in
education\citep{graceSigcse13}.  While Grace is a dynamic language at
its core, we want to have the option to teach students about types,
and so Grace supports type annotations which may be checked either
statically or dynamically to give students feedback on whether their
type annotations are correct.  We do not want students to remove
types, however, if they discover that types induce a run-time
overhead.

Additionally, we are currently maintaining three different
implementations to support a variety of educational settings
(web browsers, .NET, and JVM),
which means a typing approach for Grace ideally requires
only small changes to keep these implementations as consistent as
possible.

In this paper we illustrate that using an optimizing virtual machine allows dynamic
checks of shallow structural types with low overhead and relatively low
implementation effort. These checks are inserted na\"ively based on local
annotations and checked eagerly when control flow reaches them: 
whenever an annotated method is called or
an annotated variable is read from or assigned to,
we check types dynamically and
terminate the program with a type error if the check fails.
Despite this simplistic approach, a just-in-time compiler can
elminate the redundant checks---%
removing almost all of the checking overhead,
resulting in
a performance profile aligned with untyped code.

We evaluate this approach with Moth,
a Grace implementation on top of Truffle
and the Graal just-in-time compiler\citep{Wurthinger2013,Wurthinger:2017:PPE}.
Inspired by \citet{Richards2017} and \citet{Bauman2017},
our implementation conflates types
with information about the dynamic object structure 
(maps\citep{Self} or object shapes\citep{woss2014object}), 
which allows the just-in-time compiler
to reduce redundancy between checking structure
and checking types; and consequently, most of the overhead 
that results from type checking is eliminated.

% In this paper, we present a \emph{responds as expected} type-checking system for the Grace programming language. Through \cref{sec:method} we will describe how we extending \SOMns to support Grace and provide support for the simple type checks and, later in \cref{sec:evaluation}, we present two suites of benchmarks that demonstrate our system can offer these checks with little overhead in terms of execution time.

% While we having only taken the first step in rousing Grace's system, our implementation and the results presented are a promising step toward supporting Grace's gradual-structural type system and, ultimately, refuting the notion that gradually typed languages are dead.

The contributions of this paper are:

\begin{itemize}
\item demonstrating that VM optimisations enable
        dynamic checks of shallow structural types with low
        performance cost
\item an implementation approach that requires
      only small changes to existing abstract-syntax-tree interpreters
\item an evaluation based on classic benchmarks
      and benchmarks from the literature on gradual typing
\end{itemize}

%!TEX root = ../latex/paper.tex

\section{Background}
\label{sec:background}

This section details our motivation and
discussed the technical background for our implementation.


% \subsection{Stronger Typing for Dynamically Typed Languages}
%
% \kjx{No idea why we need this subsection}
%
% \begin{cnote}
% - the dynamic typing discussion as in the outline
% \end{cnote}

\subsection{The Grace Programming Language}
\label{ssec:grace}

% DONE \sm{feels a little verbose, if we are short on space, I'd try to be more concise in the first paragraph}

% \begin{cnote}
%   integrate:
%   The notion of message sending suggests that one object can request that another reacts to a message,
%   in which case the receiver of that message can decide how to respond.
% \end{cnote}
%\kjx{done}

We are designing Grace, an object-oriented, imperative, educational programming
language, with a focus on introductory programming
courses, but also for more advanced study and research\citep{graceOnward12,graceSigcse13}.
While Grace's syntax draws
from the so-called ``curly bracket'' tradition of C, Java, and
JavaScript (with a side order of Pascal) the structure of the language
is in many ways closer to Smalltalk (thus Self and Ruby): 
all computation is via dynamically dispatched  ``method requests''
where the object receiving the request decides which code to run;
method names
have multiple parts; blocks (lambdas) are used for control flow; and
returns within lambdas are ``non-local'', returning to the method
activation in which the block is instantiated\citep{bluebook}.  In
other ways, Grace is closer to Javascript than Smalltalk: Grace
objects can be created from object literals, rather than by
instantiating classes\citep{Black2007-emeraldHOPL,JonesECOOP2016} and
objects and classes can be deeply nested within each 
other\citep{betabook}.  

Critically, Grace's declarations and methods' arguments
and results can be annotated with types, and those types can be  checked
either statically or dynamically. This means the type system is
optional or ``pluggable'' \citep{GiladPluggable2004} (removing explicit
type annotations should not affect the semantics of a correct
program \citep{Siek2015}) and gradual (the type system
includes a distinguished ``Unknown'' type that reverts to fully 
dynamic type checking).

As an educational language\citep{panel}, absolute performance of an
implementation is less important than the performance profile---the
way language features affect performance.  Increasing absolute
performance by several orders of magnitude could let students run
larger examples---analyzing billions rather than millions of data
points, wayfinding within a city rather than a village, raytracing
higher resolution images a little quicker.  On the other hand, issues
with a language's performance profile could mean the students will
``learn the wrong things''.  If e.g.\ a languages' built-in cons lists
were faster than arrays or hash-tables, students cannot learn the
performance benefits of more complex data structures. In the case of
Grace, all extant implementations have the unfortunate property that
adding type declarations to a program makes that program run slower---teaching students that \emph{removing} type declarations is an
effective optimization technique.
Furthermore, this property is shared
by other optionally typed languages including Groovy\citep{Muehlboeck2017} and 
Dart's checked mode\citep{dartbook}.

% \citep{Muehlboeck2017}.\sm{I am not sure that, as it is phrased, is actually true.
% TypeScript does not have types at run time, they are completely
% erased, Dart}\kjx{true of dart 1 checked mode}\sm{I read over Groovy, doesn't seem to do dynamic type checking, but uses types for code generation}
% \sm{how about: A prominent example is Dart 1, which uses dynamic type checking in its checked development mode.}

% \begin{cnote}
% Nom\citep{Muehlboeck2017}
% all the gradual systems
% \end{cnote}

\subsection{Moth: Grace on Graal and Truffle}
\label{ssec:moth}

Implementing dynamic languages as state-of-the-art virtual machines
can require enormous engineering efforts.
Meta-compilation approaches\citep{Marr:2015:MTPE}
such as RPython\citep{Bolz:2009:TMP,Bolz:2013:IMT}
and GraalVM\citep{Wurthinger2013,Wurthinger:2017:PPE}
reduce the necessary work dramatically,
because they allow language implementers to leverage existing VMs
and their support for just-in-time compilation and garbage collection.

To leverage this infrastructure, we developed an interpreter for Grace,
called Moth\citep{Roberts2017}, by adapting \SOMns\citep{SOMns}.
%\citeurl{\SOMns.}{SOMns: A Newspeak for Concurrency Research}{Stefan Marr}{}{https://github.com/smarr/SOMns/}
\SOMns is a Newspeak implementation\citep{Bracha:10:NS} on top of the Truffle framework and the Graal just-in-time compiler,
which are part of the GraalVM project.
Since Newspeak and Grace are related languages,
\SOMns provides a good foundation for a new Grace implementation,
allowing us to reach the performance of V8,
Google's JavaScript implementation
(cf. \cref{sec:baseline-perf} and \citet{Marr2016})
with only moderate effort.
\SOMns was changed to parse Grace code, and 
\SOMns' self-optimizing abstract-syntax-tree nodes were only slightly adapted to conform to Grace's semantics.
As a result, Moth is mostly compliant to the Grace specification.

% \ugh{One exception is that it still supports the \code{nil} value of
  % Newspeak.}\kjx{who cares}

\sm{we could say many more things here, not sure we need to though, depends a bit on what we need/rely on in late text}
\kjx{so we fix it later}

\kjx{do we need to talk any more about SOMns here?}

% While not everyone has time to create their own fully-fledged virtual, recent efforts such as PyPy \rrnote{CITE Tratt} or Truffle and Graal \cite{Wurthinger2013} are brining the power of VMs to researchers and hobbyists.

% As shown by Marr et al. \cite{Marr2016}, the peak performance of benchmarks is comparable to monolithic VM's like V8 and Crystal.

% \subsection{Gradual Typing}
%
% \sm{not sure this belongs here. it depends on how the paper is framed,
% but my impression was that we don't want to say it is gradual, so,
% this belongs perhaps better into the related work section for discussion
% after the results}
% \kjx{don't think it does}
%
% Coined by \citet{Siek2006}, later added the gradual ``guarantee''\citep{Siek2015}.
%
% \paragraph{Languages}
%
% \sm{not sure what is supposed to be discussed here.
% Seems to detailed, seems to to be long into the beginning of the paper}
%
% \begin{itemize}
%     \item Grace \cite{Boyland2014}
%     \item Typed Racket \cite{Takikawa2016}
%     \item Dart \cite{Heinze2016, Mezzetti2016}
%     \item Pycket (using some kind of PyPy-like design) \cite{Bauman2017}
%     \item Grift (seems to be the name of a compiler) \cite{Kuhlenschmidt:2018:preprint}
%     \item TypeScript on HiggsVM \cite{Richards2017}
%     \item Reticulated Python \cite{Vitousek2017, Greenman2018}
% \end{itemize}
%
% \paragraph{Different types of type systems}
%
% \sm{Also rather for related work, I think}
%
% \begin{itemize}
%     \item Nominal \cite{Muehlboeck2017}
%     \item Structural \cite{Richards2017} (and Grace, \cite{Boyland2014})
%     \item Tag (RPython) \cite{Greenman2018}
% \end{itemize}
%
% \paragraph{Other papers}
%
% \sm{no idea what this is about}
% \kjx{just random recent stuff}
% \begin{itemize}
%     \item \cite{Bloom2009}
%     \item \cite{Castagna2017}
%     \item \cite{Stulova2016}
% \end{itemize}

%!TEX root = ../latex/paper.tex

\section{Dynamic Type Checks in Grace}
\label{sec:method}

% evaluation cases

% high flexibility, no need for global constitency 


% design goals behind Grace / Moth
As described in \cref{ssec:grace},
Grace is an educational language and its performance profile
is important for supporting effective teaching. 
%
%\mwh{It seems like a \textit{remarkably terrible} idea to be proposing
%  this as a contribution, especially when it's conflated and confused
%  with the performance contributions.}
%\kjx{we wont}
%
% motivation = give students low-overhead method to check program
The core of Grace's static type system is well described elsewhere\citep{JonesECOOP2016}; here we explain how these types can be understood
dynamically, from a student's or a programmer's point of view.
%\kjx{code for: no formalism, sorry.}
Following from the design goals behind Grace,
our motivation for this work
is to provide a flexible system 
to check the consistency of the execution of their programs 
against type annotations,
but without significant impact on run-time performance.
A secondary goal is to have a design that can be implemented with
only a small set of changes to facilitate integration in existing systems.
%
% gradual typing drawback = slow
%\chg{Gradual typing systems are generally good candidates to achieve
%this goal.}{Gradual typing systems are good candidates,because they
%provide the gradual guarantee and the mechanisms for blame would be desirable.}
% Existing gradual typing systems 
% that address the goal of flexibility currently
% Unfortunately, they do not yet have the desired performance properties\citep{Takikawa2016,Vitousek2017,Muehlboeck2017,Bauman2017,Richards2017,Greenman2018}.
% Specifically, they could encourage students to remove types to improve performance.
%


These goals are shared with much of the other work on gradual type
systems, but our context leads to some different choices. First, so
that students can see concrete examples of type errors, they should be
able to run their programs even if those programs are not type-correct%
---\ie Grace's static type checking is optional, and so an
implementation cannot depend on the correctness of a program's type
annotations. Second, while checking Grace's type annotations
statically checking may be optional, checking them dynamically should
not be: any value that flows into a variable, argument, or result
annotated with a type must conform to that type.  Third, adding type
annotations should not degrade a program's performance, or rather,
students should not be encouraged to improve performance by removing
type annotations.  Unfortunately, existing gradual type
implementations do not meet these goals, particularly the regarding
performance: thus the ongoing debate about whether gradual typing is
alive, dead, or in some state in
between\citep{Takikawa2016,Vitousek2017,Muehlboeck2017,Bauman2017,Richards2017,Greenman2018}.
% Specifically, they could encourage students to remove types to improve performance.
\mwh{This paragraph seems confused to me. They're good, but also they're bad. What is a ``gradual typing system'' here, and how does it relate to ``gradual typing'' and ``type systems'' as separate concepts?}
\sm{the above change makes the desirable things about gradual typing more concrete. wrt ``gradual typing'' and ``type systems''. one is a general idea, and a concrete realization is a ``system''. I don't know whether this is how typing people talk. but this is how I see the distinction.
we don't want to criticize the abstract idea, but concrete systems.}
\kjx{I've rewritten this bit, particularly the last paragraph.}
\kjx{I'm not sure if the goals should go here, or in the intro?}
\kjx{now we have two different lists of goals here}
\kjx{I'm more than half temped to just delete all of 3.0 --- but I've left
it for now.}

% % KJX likes this bit, wrote it below, doesn't seem to belong there
% % 
% % Part of the philosophy of Grace is that the language should not force
% % students to annotate programs with types until they are ready, so that
% % teachers can choose whether to introduce types, early, late, or even
% % not a all.  Assuming 



% % summary of goals
% Instead, the design behind our type system is focused
% on maximizing performance and flexibility while
% preserving the ability for programmers to check the execution
% of their programs.

% % how we address flexible
% To address the goal of flexibility we propose a system that is optional,
% which enables the programmer to benefit from adding checks
% to be performed at run time,
% without the burden of needing to fully type each program.

% how we address min overhead (avoid blame)
% that enables the executed behavior of the program to be checked against
% the documented types without the
% (which leads to significant overhead in previous work).


\subsection{Design}

\kjx{I worked though this mostly doing terminology,
and perhaps moving from claims to descriptions.
Does this make sense? is it any better?}

% JN killed
% To achieve the consistency checking with minimal run-time overhead,
% we propose a system with shallow structural type checks,
% which provide useful feedback without requiring a blame mechanism
% as in typical gradually typed approaches.

% Message sending
% As we described in \cref{ssec:grace}, Grace is message-sending-based language.

Our static type checks for Grace are designed to be simple,
straightforward, and hopefully easy for students to understand:
\begin{itemize}
  \item types are shallow interfaces
  \item optional type annotations are checked at run time
  \item failing run-time type checks terminate execution
\end{itemize}

% Together these properties enable a flexible system
% with minimal overhead
% that developers can use to verify that programs
% execute with values that have the capabilities
% documented through their annotations
% while avoiding the burden of fully typing their program.

We illustrate how the type checks work in practice
in the context of an exercise where a student is developing
a program to record information about vehicles.
%
Grace is structurally typed\citep{graceOnward12}:
an object implements a type whenever it
implements all the methods required by a 
type (as in Go) rather than requiring types to be declared
explicitly. 
A type expresses the requests
an object can respond to, for example whether a particular accessor is
available,  rather than a location in an explicit inheritance
hierarchy.

For example, our student can begin developing their vehicle
application by defining an object intended to represent a car
(\cref{lst:car-reg}, \cref{ex:object}) and write a method that, given
the car object, prints out its registration number (\cref{ex:method}).

\begin{lstlisting}[caption={The start of a simple program for tracking vehicle information.},label=lst:car-reg,escapechar=|,columns=flexible]
def car = object {|\label{ex:object}|
    var registration is public := "JO3553"
}

method printRegistration(v) {|\label{ex:method}|
    print "Registration: {v.registration}"
}
\end{lstlisting}

Next, the student could decide to ensure that any object passed to the
\code{printRegistration} method will respond to the
\code{registration} request.  To get this support, the student first
defines the structural type \code{Vehicle}\citep{theCleanVehicle}
naming just that method \cref{ex:vehicle}
\cref{ex:adding-type:vehicle}, and then annotates the
\code{printRegistration} method's argument with that type
(\cref{ex:vehicle} \cref{ex:adding-type}).  This ensures the student
will be alerted as soon as a value that does not conform to this
expected type is passed into the \code{printRegistration} method,
rather than that method crashing somewhere in the middle of the
implementation of the call to \code{print}.


\begin{lstlisting}[label={ex:vehicle},caption={Adding a type annotation to a method parameter.},escapechar=|,columns=flexible]
type Vehicle = interface = { |\label{ex:adding-type:vehicle}|
    registration    
}

method printRegistration(v: Vehicle) { |\label{ex:adding-type}|
    print "Registration: {v.registration}"
}
\end{lstlisting}

% \sm{you already said the following}
% By adding the type annotation to the program,
% the student can now be sure the \code{getReg} method
% will only be invoked when its arguments can respond to this message.

%\paragraph{Flexibility.}
%\paragraph{Semnatics}

While Grace's static type system supports full static type checking\citep{graceOnward12}, Grace's dynamic type tests are \emph{shallow},
that is, they check only for the presence of methods in an object,
rather than also checking conformance of argument and result types.
This is to ensure that the presence or absence of type annotations
does not affect the execution of a program, for the reason originally
outlined by \citet{Boyland2014}, thus maintaining a version of
the gradual guarantee.
The resulting semantics are more-or-less equivalent to type-tag soundness\citep{Greenman2018}---the difference being that where type-tag
soundness supports shallow \emph{nominal} type checks, we support
shallow \emph{structural} type checks.

% The notion of our type system being shallow means 
% that members of a type are untyped.
% In particular, our design does not have information on parameter types
% and return types for types' members.
% This design implies that blame tracking is not needed,\mwh{Um.}
% because the types are less detailed than in other systems.\mwh{I would hope that a competent reviewer would't fall for this\ldots}\sm{ok, what can we do about this? what's the precise issue here?
% The idea is that we do not need blame, because we do not have the type casts
% from blame-supported gradual systems.
% The difference is, I think that there is nothing we can not check immediately,
% and error as soon as there is a type.
% I believe the examples are usually higher-order functions/blocks.
% We only check the arity. So, there is no type cast wrapping necessary,
% when we pass it into a method that would expect another type of block arguments.
% (because passing in should not error, the error should only happen when
% the block is used in these systems, right?)}
% One reason for this design is that it avoids the overhead incurred
% by tracking type assumptions for precise blame attribution.



% Another aspect of this design is
% that our shallow approach to types
% allows for more flexible use 
% without requiring type parameters, \ie, generics.
For example, in \cref{ex:complex}, the student
develops two methods to create cars 
(\cref{ex:personal-car,ex:government-car}),
in which both methods are typed and return objects that conform to
an expanded \code{Vehicle} type (\cref{ex:new-vehicle}).
Note that each version of the \code{registerTo} method
declares a different type for its parameter
(\cref{ex:personal-car:registerTo,ex:government-car:registerTo}).
% Depending on its semantics,
% a less-shallow type checking system could throw an error
% due to the inconsistency.
% In contrast
% our approach allows the student to execute the program 
% despite its inconsistency, 
% while preserving the guarantee that any value found to be
% inconsistent with the annotation will result in
% a termination by a typing error.
When the student runs this program, both \code{personal\-Car} and
\code{governmentCar} can be assigned to \code{Vehicle} because that
check considers only that the vehicle has a \code{registerTo} method,
but not the required argument type of that method.
At \cref{ex:invoke-register-to} the student can attempt to register a
government car to a person: only when the method is invoked
(\cref{ex:government-car:registerTo}) the dynamic type test on the
argument will fail (the object that is passed in is not a
\code{Department}) even so the body of the \code{registerTo} method
does not rely on the \code{code} method that the \code{Department}
annotation requires of the argument.

% \sm{don't say anything about stuff we don't do. this is not really necessary,
% and I don't think this is necessarily correct, depending on specific type systems}
% A full structural type-checking system would alert the student to
% the inconsistency described above statically,
% perhaps displaying an error in their idea.
% While the alert is correct in that their is an inconsistency,
% such a system is less flexible.

% \sm{I would just say: possible inconsistency between elements can be approached
% step wise, when necessary as the sophistication and completeness of the program improves}
% Our system offers a higher level of flexibility in that the student
% may still execute the program,
% which may indeed be only partially developed and remain globally inconsistent.
% The student may be satisfied that a particular test passes successfully, and
% is then free to address the inconsistency in later development.


\begin{lstlisting}[caption={A program in development with a well-typed execution.},escapechar=|,label={ex:complex},float,floatplacement=htb,columns=flexible]
type Vehicle = interface = { |\label{ex:new-vehicle}|
    registration
    registerTo(_)
}

type Person = interface { name }
type Department = interface { code }

var personalCar : Vehicle := |\label{ex:personal-car}|
  object {
    var registration is public := "DLS018"
    method registerTo(p: Person) {|\label{ex:personal-car:registerTo}|
      print "{p.name} registers {self}"
    } 
  }

var governmentCar : Vehicle := |\label{ex:government-car}|
  object {
    var registration is public := "FKD218"
    method registerTo(d: Department) { |\label{ex:government-car:registerTo}|
      print "some department {self}"
    }
  }

governmentCar.registerTo( |\label{ex:invoke-register-to}|
  object {
    var name is public := "Richard"
  }
)
\end{lstlisting}


%\paragraph{Termination by Type Error.}
\label{sec:term-type-error}

When executing a program without types, there are three possible outcomes.
Either the program (1) terminates successfully,
(2) terminates with an exception, or 
(3) the execution diverges, \ie, it does not terminate.
Using our approach, a
fourth outcome is possible: termination with a type error.
Our implementation checks  every type annotation
on the values of arguments before invoking a method, 
on the value returned by a method when it returns, and
before any assignment to and read from a variable 
(either local to a method or belonging to an object). 

The checks are performed eagerly%
---as soon as they are encountered during execution---%
and cause the execution to terminate with a typing error 
when a value fails to implement its expected type.

% % kjx deleted this summary because it doesn't add anything.
% % 
% % % Summarize the design section
% % \paragraph{Summary}
% % Our type checking approach enables developers to express
% % the capabilities of objects throughout different components of 
% % their programs.
% % Our representation of types is shallow,
% % in that a type expresses only the set of members an object
% % should offer, while excluding any further typing information. 
% % The shallow design enables a flexible use of structural types without
% % requiring type parameters.
% % We abort program execution as soon as
% % a type annotation is inconsistent with a concrete value. 
% % With these design choices, our system offers a
% % mechanism to check for well-typed executions%
% % ---rather than well-typed programs---%
% % without negatively affecting the performance profile.
% % % This both lowers overhead for the developer and,
% % % helps us to avoid the overhead of tracking blame
% % % as seen among previous works.

\subsection{Implementation} 
\label{ssec:implementation} 

%This section gives an overview of a possible implementation
%based on an abstract-syntax-tree (AST) interpreter.

We have implemented shallow dynamic structural type checks 
by extending the Moth abstract-syntax-tree (AST) interpreter for
Grace (\cref{ssec:moth}).
%
% We developed our implementation as an extension to Moth.
% As described earlier in \cref{ssec:moth},
% Moth is an AST-based interpreter on top of the Graal VM.
%
% It is optimizes itself based on for instance run-time type information.
%
%
%Based on \cref{sec:term-type-error},
Our approach needs to check types of values at run-time:

\begin{itemize}
\item the values of arguments are checked, after a method is requested, 
      but before the body of the message is executed,
\item the value returned by a method is checked after its body is executed, and
\item the value of variables are checked
      whenever assigned to or read from by user code.
\end{itemize}

% \kjxdone{why both reading and writing?  answer - doesn't matter?}

One of the goals for our approach to dynamic typing was to keep
the necessary changes to an existing implementation small,
while enabling optimization in highly efficient language runtimes.
%
In an AST interpreter, we can implement this approach by attaching the
checks to the relevant AST nodes: the expected types for the argument
and return values can be included with the node for requesting a
method, and the expected type for a variable can be attached to the
nodes for reading from and writing to a method.  In practice, we
encapsulate the logic of the check within a new type-checking AST
node.  Moth's front end was adapted to parse and record type
annotations and attach instances of this checking node as children of the
existing method, variable read, and variable write nodes.


% The check node is detailed in \cref{ssec:optimization} to discuss
% relevant optimizations.

%

The check node uses the internal representation of a Grace type
(cf. \cref{ex:type}, \cref{ex:type:check}) to test whether an observed
object conforms to it. 
% These \code{Type} objects are created by
% Grace \code{interface} expressions, and also help
% to support Grace's pattern matching facilities \cite{gracePatternsDLS12}.
An object satisfies a type, if all members required by the type are provided
by the object (\cref{ex:type:satisfied}).


\begin{lstlisting}[label={ex:type},escapechar=|,caption={Sketch of a \code{Type} in our system and its \code{check()} semantics.},float,floatplacement=htb,columns=flexible]
class Type:
  def init(members):
    self._members = members

  def is_satisfied_by(other: Type): |\label{ex:type:satisfied}|
    for m in self._members:
      if not in other._members:
        return False
    return True

  def check(obj: Object):
    t = get_type(obj)
    return self.is_satisfied_by(t) |\label{ex:type:check}|
\end{lstlisting}


\subsection{Optimization}
\label{ssec:optimization}

There are two aspects to our implementation that are critical for a minimal overhead solution:

\begin{itemize}
  \item specialized executions of the type checking node, along with guards to protect these specialized version, and
  \item a matrix to cache sub-typing relationships to eliminate redundant executions.
\end{itemize}
 
%Here we discuss each of the aspects in more detail.

\begin{lstlisting}[label={ex:typenode},escapechar=|,caption={An illustration of the type checking node that support type checking},float,floatplacement=htbp,columns=flexible]
class TypeCheckNode(Node):

  expected: Type
  record: Matrix

  @Spec(static_guard=expected.check(obj))
  def check(obj: Number): |\label{ex:typenode:number}|
    pass

  @Spec(static_guard=expected.check(obj))
  def check(obj: String): |\label{ex:typenode:string}|
    pass

  ...

  @Spec(
      guard=obj.shape==cached_shape,
      static_guard=expected.check(obj))
  def check(obj: Object, @Cached(obj.shape) cached_shape: Shape): |\label{ex:typenode:object}|
    pass
  
  @Fallback
  def check_generic(obj: Any): |\label{ex:typenode:generic}|
    T = get_type(obj)
    
    if record[T, expected] is unknown: |\label{ex:typenode:matrix}|
      record[T, expected] =
          T.is_subtype_of(expected)

    if not record[T, expected]:
      raise TypeError(
          "{obj} doesn't implement {expected}")
\end{lstlisting}

The first performance-critical aspect to our implementation
is the optimization of the type checking node.
We rely on Truffle and its TruffleDSL\citep{humer2014domainspecific}.
This means, we provide a number of special cases,
which are selected during execution based on the concrete observed
kinds of objects.
A sketch of our type checking node using a pseudo-code version of the DSL
is given in \cref{ex:typenode}.
A simple optimization is for well known types such as
numbers (\cref{ex:typenode:number}) or strings (\cref{ex:typenode:string}).
The methods annotated with \code{@Spec} (shorthand for \code{@Specialization})
correspond to possible states in a state machine that is generated by the
TruffleDSL.
Thus, if a check node observes a number or a string,
it will check on the first execution only that the expected type,
\ie, the one defined by some type annotation,
is satisfied by the object by using a \code{static\_guard}.
If this is the case, the DSL will activate this state.
For just-in-time compilation, only the activated states and their normal guards are considered.
A \code{static\_guard} is not included in the optimized code.
If a check fails, or no specialization matches, the fallback,
\ie, \code{check\_generic} is selected (\cref{ex:typenode:generic}),
which may raise a type error.

For generic objects, we rely on the specialization on \cref{ex:typenode:object}.
It checks that the object satisfies the expected type.
If that is the case, it reads the shape of the object at specialization time,
and caches it for later comparisons.
Thus, during normal execution,
we only need to read the shape of the object, compare it to the cached one
with a simple reference comparison.
If the shapes are the same, we can assume the type check passed successfully.

The other performance-critical aspect to our implementation
is the use of a matrix-based record to cache sub-typing relationships.
The matrix compares types against types,
featuring all known types along the columns and the same types again along the rows.
A cell in the table corresponds to a sub-typing relationship:
does the type corresponding to the row implement
the type corresponding to the column?
All cells in the matrix begin as unknown and as 
encountered in a check during execution
we populate the table.
If a particular relationship has been computed before
we can skip the check and instead recall the previously-computed value.
Using this table we are able to eliminate the redundancy of evaluating
the same type to type relationships across different checks in the program. To reduce redundancy further we also unify types.
Our unification of types is similar to Java's string interning; 
during the construction of a type we first check to see if the same
set of members is expressed by a previously-created type and, if so,
we avoid creating the new instance and provide the existing one.

Together the self-specializing type check node and the matrix-based record
ensure that our implementation eliminates redundancy, and
consequently, we are able to minimize the run-time overhead of our system. 

% The Graal compiler 
% Moth needs to query the shape of arguments to evaluate them.
% Moth then needs to query the shape again to run the specializations' guards.
% Once profiled Graal can remove this redundancy by eliminating
% reusing the first query.
% During this compilation Graal also employs other optimizations 
% -- sub-expression elimination, inlining, propagation and folding --\tabularnewline
% and, as we demonstrate next in \cref{sec:evaluation},
% produces a compiled version of the method with minimal overhead incurred from
% the type check.

%!TEX root = ../latex/paper.tex

\section{Evaluation}
\label{sec:evaluation}

\newcommand{\NumIterationsAll}{1000\xspace}
\newcommand{\NumIterationsHiggs}{100\xspace}


To evaluate our approach to dynamic type checking,
we first establish the baseline performance of Moth
compared to Java and JavaScript,
and then assess the impact of the type checks themselves.

\subsection{Methodology and Setup}

To account for the complex warmup behavior
of modern system\citep{Barrett:2017:VMW},
we run each benchmark for \NumIterationsAll iterations in the same
VM invocation.\footnote{
For the Higgs VM, we only use \NumIterationsHiggs iterations,
because of its lower performance.
This is sufficient since its compilation approach induces less variation
and leads to more stable measurements.}
Afterwards, we inspected the run-time plots over the iterations
and manually determined a cutoff of \WarmupCutOff iterations for warmup,
\ie, we discard iterations with signs of compilation.
In this work, we do not consider startup performance,
because we want to assess the impact on dynamic type checks
on the best possible performance.
All reported averages use the geometric mean since they aggregate ratios.

% Yuria
%  - Ubuntu 16.04.4, Kernel 3.13
%  - 24 hyperthreads
%  - Intel Xeon E5-2620 v3 2.40GHz
% Graal 0.33 Feb. 2018
All experiments were executed on a machine running Ubuntu Linux 16.04.4,
with Kernel 3.13.
The machine has two Intel Xeon E5-2620 v3 2.40GHz,
with 6 cores each, for a total of 24 hyperthreads.
We used ReBench 0.9.1, Java 1.8.0\_171, Graal 0.33 (\code{a13b888}),
Node.js 10.4, and Higgs from 9 May 2018 (\code{aa95240}).
Our experimental setup is available online to enable reproductions.\footnote{
\smtodo{add version info}
\url{https://gitlab.com/richard-roberts/moth-benchmarks/tree/dev}}


\subsection{\AWFY?}
\label{sec:baseline-perf}


\begin{figure}
	\AwfyBaseline{}
	\caption{Comparison of Java 1.8, Node.js 10.4, Higgs VM, and Moth.
  The boxplot depicts the peak-performance results for the \AWFY benchmarks,
  each benchmark normalized based on the result for Java.
  For these benchmarks, Moth is within the performance range
  of JavaScript, as implemented by Node.js,
  which makes Moth an acceptable platform for our experiments.}
	\label{fig:awfy-baseline}
\end{figure}

% - setting a base line with the AWFY benchmarks
% - comparing
%   - Java
%   - Node.js
%   - Moth
%   - Higgs
To establish the performance of Moth,
we compare it to Java and JavaScript.
For JavaScript we chose two implementations,
Node.js with V8 as well as the Higgs VM.
The Higgs VM is an interesting point of comparison,
because \citet{Richards2017} used it in their study.

We compare across languages based on the \AWFY benchmarks\citep{Marr2016},
which are designed to enable a comparison
of the effectiveness of compilers across different languages.
To this end, they use only a common set of core language elements.
While this reduces the performance relevant differences between languages,
the set of core language elements covers only common object-oriented language
features with first-class functions.
Consequently, these benchmarks are not necessarily a predictor
for application performance,
but can give a good indication for basic mechanisms such as type checking.
Furthermore, in an educational setting,
we assume that students will focus on using these basic language features
as they learn a new language.
% - arguing that Moth has state of the art performance on the given benchmarks
% - this is a reasonable foundation to make performance claims
%   that can generalize to state-of-the-art custom VMs such as V8

\Cref{fig:awfy-baseline} shows the results.
We use Java as baseline since it is the fastest language implementation
in this experiment.
We see that Node.js (V8) is about
\OverheadNodeGMeanX (min. \OverheadNodeMinX, max. \OverheadNodeMaxX)
slower than Java.
Moth is about \OverheadMothGMeanX (min. \OverheadMothMinX, max. \OverheadMothMaxX) slower than Java.
As such, its on average \OverheadMothNodeGMeanP (min. \OverheadMothNodeMinP, max. \OverheadMothNodeMaxX) slower than Node.js.
Compared to the Higgs VM, which is on these benchmarks
\OverheadHiggsGMeanX (min. \OverheadHiggsMinX, max. \OverheadHiggsMaxX) slower than Java,
Moth reaches the performance of Node.js more closely.
With these results, we argue that Moth is a suitable platform to
assess the impact of our approach to dynamic type checking,
because its performance is close enough to state-of-the-art VMs,
and run-time overhead is not hidden by slow baseline performance.


\subsection{Performance of Dynamic Type Checking}

% - benchmark selection
%  - dynamic type checking performance determined based on commonly used
%    benchmarks from the gradual typing papers
% - we are subsetting, need to explain why
%   - the subset of benchmarks that we managed to port to Grace/Moth

The performance overhead of our dynamic type checking system
is assessed based on the \AWFY benchmarks
as well as benchmarks from the gradual-typing literature.
The goal was to complement our benchmarks with additional ones that are
used for similar experiments and can be ported to Grace.
To this end, we surveyed a number of papers\citep{Takikawa2016,Vitousek2017,Muehlboeck2017,Bauman2017,Richards2017,Stulova2016,Greenman2018}
and selected benchmarks that have been used by multiple papers.
Some of these benchmarks overlapped with the \AWFY suite,
or were available in different versions.
While not always behaviorally equivalent,
we chose the \AWFY versions since we already used them to
establish the performance baseline.
The list of selected benchmarks is given in \cref{tab:gradual-benchmarks}.

The benchmarks were modified to have complete type information.
We modified Moth to report absent type information and ensured it was complete.
To assess the performance overhead of type checking,
we compare the execution of Moth with all checks disabled, \ie, the baseline version from 
\cref{sec:baseline-perf}, against an execution that has all checks enabled.


\begin{table}
\caption{Benchmarks selected from literature.}
\label{tab:gradual-benchmarks}
\begin{tabular}{l l r}
Fannkuch & \cite{Vitousek2017,Greenman2018} \\
Float & \cite{Vitousek2017,Muehlboeck2017,Greenman2018} \\
Go & \cite{Vitousek2017,Muehlboeck2017,Greenman2018} \\
NBody & \cite{Kuhlenschmidt:2018:preprint,Vitousek2017,Greenman2018} & used \cite{Marr2016} \\
Queens & \cite{Vitousek2017,Muehlboeck2017,Greenman2018} & used \cite{Marr2016} \\
PyStone & \cite{Vitousek2017,Muehlboeck2017,Greenman2018} \\
Sieve & \cite{Takikawa2016,Muehlboeck2017,Bauman2017,Richards2017} & used \cite{Marr2016} \\
Snake & \cite{Takikawa2016,Muehlboeck2017,Bauman2017,Richards2017} \\
SpectralNorm & \cite{Vitousek2017,Muehlboeck2017,Greenman2018} \\
\end{tabular}
\end{table}

\begin{figure}
	\TypingOverhead{}
	\caption{A boxplot comparing the performance of Moth without type checking
  to Moth with type checking.
  The plot depicts the run-time overhead on peak performance over
  the untyped performance. On average, dynamic type checking introduces
  an overhead of \OverheadTypingGMeanP (min. \OverheadTypingMinP, max. \OverheadTypingMaxP).}
	\label{fig:typing-overhead}
\end{figure}

The results are depicted in \cref{fig:typing-overhead}.
Overall, we see a peak-performance overhead of 
\OverheadTypingGMeanP (min. \OverheadTypingMinP, max. \OverheadTypingMaxP).

% explain maxima

The benchmark with the highest overhead of \OverheadListP is List,
which traverses a linked list and has to check the list elements individually
in a way that introduces checks that do not coincide with any shape checks
on the relevant objects that are performed in the unchecked version.
We consider this benchmark a pathological case and discuss it
in detail in \cref{sec:disc-pathological-case}.

Beside List, the highest overheads are on
Richards (\OverheadRichardsP), CD (\OverheadCDP), 
Snake (\OverheadSnakeP), and Towers (\OverheadTowersP).
Richards has one major component, also a linked list traversal,
similar to List.
Snake and Towers access primarily arrays in a way that introduces checks
that do not coincide with behavior in the unchecked version.

% explain minima

However, in some benchmarks the run time decreased; notably Permute (\OverheadPermuteP),
GraphSearch (\OverheadGraphSearchP), and Storage (\OverheadStorageP).
Permute simply creates the permutations of an array.
GraphSearch implements a page rank algorithm
and thus is primarily graph traversal.
Storage stresses the garbage collector by constructing a tree of arrays.
For these benchmarks the introduced checks seem to coincide with shape-check operations
already performed in the untyped version.
The performance improvement is possibly caused by having checks earlier,
which enables the compiler to more aggressively move them out of loops.
Another reason could simply be that the extra checks shift the boundaries
of compilation units.
In such cases, checks might not be eliminated completely,
but the shifted boundary between compilation units might mean that
objects do not need to be materialized and thus do not need to be allocated,
or simply that the generated native code interacts better with
the instruction cache of the processor.
Such shifts in performance of about 10\%
are somewhat common for highly optimizing just-in-time compilers.\mwh{cite?}


% discuss optimization techniques?
% discuss specializations?


% \begin{cnote}
%   - which experiments?
%
%   - fully types
%   - no types
%   - no types on fields
%   - all run on moth
%
% \end{cnote}

\subsection{Changes to Moth}

One of our goals, outlined earlier \cref{sec:method}, a secondary
goal of our design was to enable the implementation of our approach to be
realized with few changes to the underlying interpreter. 
Providing the system with few changes is important to us because
we maintain three different Grace implementations%
---a C\# based interpreter, a self-hosting compiler, and a high performance implementation---%
that support teaching in different contexts.
When able to be provided with fewer changes, 
we ensure that each implementation can provide the type checking
in a uniform way.

By examining the history of changes maintained by our version control, 
we estimate that implemented our dynamic checking for Moth with
slightly more than 549 new lines and 59 changed lines. 
The majority of new lines correspond to the implementation of 
new modules for the type class (179 lines) and 
the self-specializing type checking node (139 lines),
along with changes to the parsing (115 new lines, 14 lines changes)
and AST nodes (116 new lines, 45 lines changes) to extract and store
typing information.

% New Classes
% -----------
% TypeCheckNode               (139 new lines)
% SomStructuralType           (179 new lines) = 318 new

% Parsing
% -------
% returnType                  ( 18 new lines)  
% typeFor                     ( 31 new lines)
% typesForParameters          ( 27 new lines)
% typesForLocals              ( 11 new lines)
% parseTypeLiteral            ( 19 new lines)
% invoke parseTypeLiteral     (  6 new lines)
% adding return type          (  3 new lines)
% types for declarations      (  4 changed lines)
% adding type to parameters   (  4 changed lines)
% adding type locals          (  4 changed lines)
% adding type to variable     (  2 changed lines) = 115 new, 14 changed

% AST
% ---
% Storage location            ( 20 changed lines)
% local variable read/write   ( 28 new lines,  1 changed)
% add type to slot read/write (  4 new lines,  2 changed)
% cached TX slot read/write   ( 14 new lines, 21 changed)
% dispatch                    ( 31 new lines,  1 changed)
% class factory getTyoe       ( 39 new lines)     = 116 new, 45 changed


%!TEX root = ../latex/paper.tex

\section{Discussion}

\begin{cnote}
- pathological case
  var a = obj
  while (...) \{
    a = a.next
  \}
  in this case, we introduce a new type check that does not coincide with any
  existing checks (shape check on method dispatch, etc)
  consequently, we see overhead that can be up to XX\% (cf. sec. XX).
\end{cnote}


%!TEX root = ../latex/paper.tex

\section{Related Work}
\label{sec:related-work}

Although syntaxes for type annotations in dynamic languages go back at
least as far as Lisp\citep{cltl2}, the first attempts at adding a
comprehensive static type system to a dynamically typed
language involved 
Smalltalk\citep{RalphJohnson1986}, with the first practical system
being Bracha's Strongtalk\citep{strongtalk}. Strongtalk
(independently replicated for Ruby\citep{DBRuby09}) provided a
powerful and flexible static type system, where crucially, the system
was \emph{optional} (also known as pluggable
\cite{GiladPluggable2004}). Programmers could run the static checker
over their Smalltalk code (or not); either way the type annotations
had no impact whatsoever of the semantics of the underlying Smalltalk
program.

\citet{Siek2006} introduced the term ``gradual typing''
 to describe the logical extension of this scheme: a
dynamic language with type annotations that could, if necessary, be
checked at runtime. \citeauthor{Siek2006} build on earlier complementary work extending fully statically typed languages with a ``\texttt{DYNAMIC}''
type---\citet{AbadiTOPLAS1991} is an
important early attempt
and also surveys previous work. Revived practical adoption of dynamic
languages generated revived research interest, leading to the
formulation of the ``gradual guarantee''\citep{Siek2006,Siek2015} to characterize sound
gradual type systems: removing type annotations should not change the
semantics of a correct program, drawing on the Boyland's critical
insight that, of course, such a guarantee must by its nature forbid
code that can depend on the presence or absence of type declarations 
elsewhere in the program\citep{Boyland2014}. 

Type errors in gradual (or other dynamically checked) type systems will
often be triggered by the type declarations, but often those
declarations will not be at fault---indeed in a correctly typed
program in a sound gradually typed system,  the declarations cannot be
at fault because they will have passed the static type
checker. Rather, the underlying fault must be somewhere within the
barbarian dynamically typed code \emph{trans vallum}.
Blame tracking\citep{blame2009,blameThreesomes2010,blameForAll2011} localizes these
faults by identifying 
the point in the program where the system makes an 
assumption about dynamically typed objects, so can identify the root
cause should the assumption fail.  Different semantics for blame
detect these faults slightly differently, and impose more or less
implementation overhead\citep{reticPython2014,monotonic2015,Vitousek2017}.

As with language designs, there seem to be two main implementation
strategies for languages mixing dynamic and static type checks: either
adding static checks into a dynamic language implementation, or adding
support for a dynamic types to an implementation that depends on
static types for efficiently. Racket, for example, optimizes code with
a combination of type inference and type declarations---the Racket
IDE ``optimizer coach'' goes as far as to suggest to programmers type
annotations that may improve their program's performance\citep{optimizerCoach2012}. In these implementations, values flowing
from dynamically to statically typed code must be checked at the
boundary.  Fully statically typed code needs no dynamic type checks,
and so generally performs better than dynamically typed code. Adopting
a gradual type system such as Typed Racket\citep{typedScheme08} allows
programmers declare types explicitly that can be checked statically,
removing unnecessary overhead.

On the other hand, systems such as Reticulated Python\citep{reticPython2014}, SafeTypeScript\citep{Richards2017}, and our
work, here takes the opposite approach: adding type checking into
just-in-time compiling virtual machines that do not rely on static
type declarations. These systems do not use information from type
declarations to optimize execution speed, rather the necessity to
perform (potentially repeated) dynamic type checks tends to slow
programs down, so here code with no type annotations generally
performs better than statically typed code, or rather, code with many
type annotations. In the limit, these kinds of systems may only ever
check types dynamically and may not involve a static type checker at
all. 
%% SM: removed because I still think this isn't correct
%  Several recent languages including HACK \cite{HACK}, Typescript
% \cite{typeScriptECOOP,typeScriptTyping}, Dart \cite{dartbook}, and Grace
% \cite{graceOnward12} follow this general approach.

As these systems have come to wider attention, the question of their
implementation overheads has become more prominent.  
\citet{Takikawa2016} asked ``is sound gradual typing
dead?'' based on a systematic performance measurement on Typed Racket.
The key here is their evaluation methodology, where they constructed a
number of different permutations of typed and untyped code, and
evaluated performance along the spectrum.
\citet{Bauman2017} replied to \citeauthor{Takikawa2016}'s study, but
using Pycket\citep{Pycket2015}, a tracing JIT for Racket, rather
than the standard Racket VM, although maintaining full gradually typed
Racket semantics. \citeauthor{Bauman2017} are able to demonstrate most benchmarks
with a slowdown of 2x on average over all configurations.
Note that this is not directly comparable to our system,
since typed modules do not need to do any checks at run time.
Typed Racket only needs to perform checks at boundaries between typed and untyped modules,
however, they use the same essential optimization technique that we apply,
using object shapes to encode information about gradual types.
% - argument about acceptable performance made indirectly
%   using the "CDF-based slowdown plots"
%   - allows choice of arbitrary threshold of what is acceptable overhead
%     - Pycket always better than Racket
%     - not as applicable in our case, because we don't do module-based typing
%   - use Typed Racket's approach to gradual type checking
%   - optimize it by using object shapes, as we do
%     however, because of their more sophisticated semantics,
%     the optimization does only remove part of the overhead
%     specifically, they optimize the number of objects that need to be traversed
%     to perform checks based on their contracts
%     however, they can not reduce it to a simple pointer comparison as in our
%     case, though they get the benefit of more precise checks


\citet{Muehlboeck2017} also replied to \citeauthor{Takikawa2016}, 
using a similar benchmarking methodology applied to Nom, a language
with features designed to make gradual types easier to optimize, 
demonstrating speedups as more type information is added to programs.
Their approach enables such type-driven optimizations,
but relies on a static analysis which can utilize the type information.


% \rrtodo{Are we nearly equal Greenman,
% in the sense that a reference to a type object the same as a tag?}


%% SM: Removed, I can't say anything concrete here.
%%     No idea how to compare to this system.
%%     Their base lines are just too diffent, and I have no how these systems
%%     would perform on our benchmarks.

%%%dumpted in anyway since we cite it, we should say something
%%%even if we are saying nothing
%%%
Most recently, \citet{Kuhlenschmidt:2018:preprint} employ an
ahead of time (\ie traditional, static) compiler for a custom
language called Grift and demonstrate good performance for code where more than
half of the program is annotated with types, and reasonable
performance for code without type annotations. 



Perhaps the closest to our approach are \citet{Greenman2018}
and \citet{Richards2017}.
\citeauthor{Greenman2018} describe dynamically checking ``tag-type''
soundness for Reticulated Python.
As with our work, \citeauthor{Greenman2018} check only the ``top-level'' type of an
object against a declaration, although \citeauthor{Greenman2018}'s checks are nominal
while ours are structural.
% - based on Reticulated Python
  % - Python 3-based implementation (interpretation, no compilation)
%- soundness notion and supported semantics very similar to ours
%- evaluate the performance in detail
% - generate different variants to see how the number of type checks correlate
%   with performance
% - we could possibly do the same, but expect similar results and instead
%   focused on identifying more precisely which types of type annotations
%   might relate to performance issues
%   - thus, we only tested N configurations per benchmark\sm{N should be a macro}
We refrain from a performance comparison since
Reticulated Python is an interpreter without just-in-time compilation
and thus performance tradeoffs are different.

\citet{Richards2017} take a similar implementation
approach to our work, demonstrating that key mechanisms such as object shapes
used by a VM to optimize dynamic languages can be used to eliminate most of
the overhead of dynamic type checks.
Unlike our work, Richards
implement ``full'' gradual typing with blame tracking, rather than
shallow structural checks, and do so on top of an adapted Higgs
VM.
The Higgs VM implements a baseline just-in-time compiler based on
basic-block versioning\citep{Chevalier-Boisvert:2016:ITS}.
In contrast, our implementation of dynamic checks
is built on top of the Truffle framework for the Graal VM, and reaches
performance approaching that of V8 (cf. \cref{sec:baseline-perf}).
The performance difference is of relevance here since any small constant factors
introduces into a VM with a lower baseline performance can remain hidden,
while they stand out more prominently on a faster baseline.

Overall, it is unclear whether our results confirm the ones
reported by \citet{Richards2017},
because our system is simpler.
It does not introduce the polymorphism
issues caused by accumulating cast information on object shapes,
which could be important for performance.
Considering that \citeauthor{Richards2017} report ca. 4\% overhead
on the classic Richards benchmark, while we see \OverheadRichardsP,
further work seems necessary to understand the performance implications of
their approach for a highly optimizing just-in-time compiler.

%- based on Higgs JavaScript VM
%- SafeTypeScript
%  - system of object contracts
%  - checked lazily where possible
%    i.e., when accessing an object (reading, writing fields, method invocations are preceded by field read)
%    for a function object, which may have contracts, before it is invoked
%  - as described in sec. XX, we only check the availability of methods/fields
%    in a type
%    - this has also consequences on checking higher-order functions
%      - we only check whether they take the expected number of arguments,
%        which is part of their type, but don't check types of their arguments
%        (same for normal methods)
%  - optimized by recording the object contracts as part of the shape
%  - this technique is essentially identical to ours
%    - our type information is encoded in the shape, too
%    - however, we check conformance eagerly
%      - this is as part of having semantics for Grace that provide immediate
%        feedback on correctness
%    - from a performance perspective, their approach could have an even higher
%      cost: increased degree of polymorphism (ours is not increased,
%      because we do not use shape trees.
%      each class has a single shape, shared by all instances
%      the type is based on the specified members, which are statically known)
%
%- Higgs
%  - no inlining, escape analysis, tuned GC, etc.
%  - any optimizations across basic blocks (beside type propagation)




%!TEX root = ../latex/paper.tex

\section{Conclusion}
\label{sec:conclusion}

%% Context
With the wide-spread use of dynamically typed languages,
research on efficient approaches for type checking
such languages becomes more important.
%
%% goals
%% approach to support types in a dynamic language
%% - without significant overhead
%% - simple to implement
This work proposes a new design for checking shallow structural
types at run time.
The main goal of the design was to enable an implementation that
does not introduce significant run time overhead.
A secondary goal was to be able to implement this approach for
existing language implementations with only a small amount of work.


%% proposed technique
%% - shallow structural types
The proposed approach to shallow types considers only the names of type members.
Thus, it does not consider types on these members.
With this shallow design, we enable flexibility in the sense that
globally inconsistent programs can still be executed and
checked for consistency with their annotations as far as
that particular execution goes%
--- an attribute that supports an incremental style of development
that may be more common among students learning programming---%.
% Furthermore, it enables precise feedback on type violations
% without requiring blame tracking,
% as it is usually needed for approaches to gradual typing
% at the cost of run-time performance.

%% - type checks, optimization based on object shapes
In our design types are checked on access to variables,
when assigning to method parameters, and also on return values.
The information on types is encoded as part of an object's shape,
which means that shape checks already performed in a normal dynamic
language implementation can be used to check types, too. 
Being able to tie checks to the shapes in this way is critical for 
a minimal overhead implementation of dynamic checking.

%% results
%% + context (baseline)
We evaluated this approach with Moth, an implementation of the Grace language
on top of Truffle and Graal.
Using the \AWFY benchmarks as well as a collection of benchmarks from the
gradual typing literature, we find that our approach to dynamic type checking
introduces an overhead of 
\OverheadTypingGMeanP (min. \OverheadTypingMinP, max. \OverheadTypingMaxP)
on peak performance.
Since Moth reaches the performance of a
highly optimized JavaScript VM such as V8,
we believe that these results are a good indication
for the low overhead of our approach.

%% relevance
%%   where to go from here?

%% future work
In specific cases, the overhead is still significant and requires further
research to be practical. Thus, future research should investigate how the
number of dynamic type checks can be reduced without causing
the type feedback to become too imprecise to be useful.
One approach might increase the necessary changes to a language implementation,
but avoid checking every variable read.
Another approach might further leverage Truffle's self-specialization
to propagate type requirements and avoid unnecessary checks.
%%  - investigate optimizations
%%  - might enable a full structural type system, \ie, not shallow
%%  - verify results of Richards2017 in context of highly optimizing VM
Full structural types should be further investigated, too.
We believe it should be verified that \citet{Richards2017} result generalize
to highly optimizing virtual machines,
or other optimizations for precise blame would need to be investigated.



\bibliographystyle{ACM-Reference-Format}
\bibliography{references}

\end{document}
