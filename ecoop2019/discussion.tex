%!TEX root = paper.tex

\section{Discussion}
\label{sec:discussion}

\subsection{The VM Could Not Already Know That}
\label{sec:disc-pathological-case}

% - pathological case
%   in this case, we introduce a new type check that does not coincide with any
%   existing checks (shape check on method dispatch, etc)
%   consequently, we see overhead that can be up to XX\% (cf. sec. XX).


One of the key optimizations for our work
and the work of others\citep{Richards2017,Bauman2017}
is the use of object shapes
to encode information about types
(in our case),
or type casts and assumptions
(in the case of gradually typed systems).

The general idea is that a VM will already use object shapes
for method dispatches, field accesses, and other operations on objects.
Thus any further use to also imply type information
can often be optimized away
when the compiler sees that the same checks are done
(and therefore can be combined). This is similar to 
the elimination of other side-effect-free common subexpressions.

This assumption breaks, however, when checks are introduced
that do not correspond to those that exist already.
As described in \cref{sec:method},
our approach introduces checks for reading and writing to variables.
\Cref{ex:pathological-case} gives an example of a pathological case.
It is a loop traversing a linked list.
For this example our approach 
introduces a check, for the \code{ListElement} type,
when (1) assigning to and reading from \code{elem} and
(2) when activating the \code{next} method.
The checks for reading from \code{elem} and activating the method can be
combined with the dispatch's check on object shape.
%
Unfortunately, the compiler cannot remove the check
when writing to \code{elem}, because it has no information about
what value will be returned from \code{next}, and so it needs to preserve the check to be able to trigger an error
on the assignment.
For our List benchmark, this check induces an overhead of \OverheadListP.
% TODO: reconsider after submission:
% \sm{I am not sure why this does not fold with the .next check.
% I am sure that it couldn't when we were at the whiteboard,
% but I don't see it any more}

\begin{lstlisting}[caption={Example for dynamic type checks not corresponding to existing checks.},escapechar=|,label={ex:pathological-case},float,floatplacement=htb]
var elem: ListElement := headOfList
while (...) do {
  elem := elem.next
}
\end{lstlisting}

\subsection{Optimizations}

% - possible optimization for reads
% - discuss what the effect of dropping checks on reads would be
%   -> need to change more of the primitives
%   -> don't do this a the moment
%   -> could give performance benefits (experiment?)
%   -> but would either delay errors (more imprecision)
%      or require all assignments, also from primitives to be checked

As a simplification, we currently check variable access
on both reads and writes.
This approach simplifies the implementation, because we do not need to
adapt all built-ins, \ie,
all primitive operations provided by the interpreter.
One optimization could be to avoid read checks.
A type violation can normally only occur when writing to a variable,
but not when reading.
However, to maintain the semantics, this would require us to adapt
many primitives; 
such as operations that activate blocks to check their arguments,
or that write to variables or fields.
With our current implementation
we get errors as soon as user code accesses fields,
% but not primitive operations,
which simplifies the implementation.

%   - we can use truffle's approach to specializing ASTs to propagate
%     type constraints into the code we are using, which means,
%     all type checks can be omitted for code that promises to obey
%     type constraints.
%     Idea: pass expected return type down, don't check it when getting
%      the return value. instead, us truffle approach of reverting optimizations
%      by indicating violations with an exception.

Another optimization could be to use Truffle's approach to 
self-specialization\citep{Wurthinger:2012:SelfOptAST}
and propagate type information to avoid redundant checks.
At the moment, Truffle interpreters typically use self-specialization to 
specialize the AST to avoid boxing of primitive types.
This is done by speculating that some subtree always returns the expected type.
If this is not the case, the return value of the subtree is going to be
propagated via an exception, which is caught and triggers respecialization.
This idea could possibly be used to encode higher-level type information for
return values, too.
This could be used to remove redundant checks in the interpreter
by simply discovering at run time that whole subexpressions conform to the
type annotations.

\subsection{Threats to Validity}

%\kjx{this is a first draft, people should fix / check it!}

This work is subject to many of the threats to validity common to
evaluations of experimental language implementations.  Our underlying
implementation may contain undetected bugs that affect the semantics
or performance of the gradual typing checks, affecting construct
validity --- we may not have implemented what we think we have. Given
that, our benchmarking harness run on the same implementation is
subject to the same risks, thus also affecting internal validity ---
we may not be measuring that implementation correctly.  Moth is built
on the Truffle and Graal toolchain, so we expect external validity
there at least --- we expect the results would transfer to other Graal
VMs doing similar AST-based optimizations.  We have less external
validity regarding other kinds of VMs (such as VMs specialized to 
particular languages, or VMs built via meta-tracing rather than partial evaluation). 
Nevertheless, we expect our results should be transferable
as we rely on common techniques.

Finally, because we are working in Grace, it is less obvious that our
results generalize to other gradually typed-languages. We have worked
to ensure e.g.\ our benchmarks do not depend on any features of Grace
that are not common in other gradually-typed object-oriented
languages, but as Grace lacks a large corpus of programs the
benchmarks are necessarily artificial, and it is not clear how the
results would transfer to the kinds of programs actually written in
practice. The advantage of Grace (and Moth) for this research is
that their relative simplicity means we have been able to build an
implementation that features competitive performance with significantly less
effort than would be required for larger and more complex languages.
On the other hand, more effort on optimisations could well lead to
even better performance.
