%!TEX root = paper.tex

\section{Discussion}
\label{sec:discussion}

\subsection{The VM Could Not Already Know That}
\label{sec:disc-pathological-case}

% - pathological case
%   in this case, we introduce a new type check that does not coincide with any
%   existing checks (shape check on method dispatch, etc)
%   consequently, we see overhead that can be up to XX\% (cf. sec. XX).


One of the key optimizations for our work
and the work of others\citep{Bauman2017,Richards2017}
is the use of object shapes
to encode information about types
(in our case),
or type casts and assumptions
(in the case of gradually typed systems).
%
The general idea is that a VM will already use object shapes
for method dispatches, field accesses, and other operations on objects.
Thus any further use to also imply type information
can often be optimized away
when the compiler sees that the same checks are done,
and therefore can be combined
by optimizations such as common subexpression elimination.

This assumption breaks, however, when checks are introduced
that do not correspond to those that exist already.
As described in \cref{sec:method},
our approach introduces checks for reading from and writing to variables.
\Cref{ex:pathological-case} gives an example of a pathological case.
It is a loop traversing a linked list.
For this example our approach 
introduces a check, for the \code{ListElement} type,
when (1) assigning to and reading from \code{elem} and
(2) when activating the \code{next} method.
The checks for reading from \code{elem} and activating the method can be
combined with the dispatch's check on object shape.
%
Unfortunately, the compiler cannot remove the check
when writing to \code{elem}, because it has no information about
what value will be returned from \code{next}, and so it needs to preserve the check to be able to trigger an error
on the assignment.
For our List benchmark, this check induces an overhead of \OverheadListP.

Compiler optimizations such as inlining are also insufficient
for this particular case, because there are no guarantees about what
\code{elem} does to implement \code{next}.
The \code{next} method of a specific kind of \code{ListElement} may even have
a type annotation for a return value.
The best Graal can do in this example is to combine the check for the
return value with the one writing to \code{elem}.

\begin{lstlisting}[caption={Example for dynamic type checks not corresponding to existing checks.},%
  escapechar=|,label={ex:pathological-case},%
  float,floatplacement=htb]
type ListElement = interface {
  next
}

var elem: ListElement := headOfList
while (...) do {
  elem := elem.next
}
\end{lstlisting}

Since the example shows a somewhat generic data structure,
there is the question of whether the issue applies to other data structures as
well.
Our benchmarks use a range of data structures including hash maps, sets,
and vectors, none of which show the issue, because in more complex programs
the chance of already having a check there is high,
and cases were there has not been one before seem to be rare --- although
one can always consider additional optimizations
to eliminate further checks.
For generic data structures, storage strategies\citep{Bolz2013}
could be used to encode type information about elements.
This would allow the VM to check only once before a loop,
and the loop could then rely on that check for guarantees
about the elements of the data structure.


\subsection{Optimizations}
\label{sec:discussion-opt}

% - possible optimization for reads
% - discuss what the effect of dropping checks on reads would be
%   -> need to change more of the primitives
%   -> don't do this a the moment
%   -> could give performance benefits (experiment?)
%   -> but would either delay errors (more imprecision)
%      or require all assignments, also from primitives to be checked

\subparagraph{Read and Write Checks.}
\label{sec:discussion-read-write}

As a simplification, we currently check variable access
on both reads and writes.
This approach simplifies the implementation, because we do not need to
adapt all built-ins, \ie,
all primitive operations provided by the interpreter.
One optimization could be to avoid read checks.
A type violation can normally only occur when writing to a variable,
but not when reading.
However, to maintain the semantics, this would require us to adapt
many primitives.
Examples for primitives are operations that activate blocks,
which need to check their arguments or return values
as well as any primitives that write to variables or fields.
Given the number of primitives, this is error prone
and incompleteness would result in missing type checks.

By checking reads and writes in a few well defined locations,
we get errors as soon as user code accesses fields and variables.
Moreover, only a small set of locations required changes to the code,
which simplified the implementation.
Given the good
results (cf. \cref{sec:eval-opt-perf,sec:eval-changes}), we decided to keep read checks, because it is a more uniform and maintainable
approach for an academic project.

%   - we can use truffle's approach to specializing ASTs to propagate
%     type constraints into the code we are using, which means,
%     all type checks can be omitted for code that promises to obey
%     type constraints.
%     Idea: pass expected return type down, don't check it when getting
%      the return value. instead, us truffle approach of reverting optimizations
%      by indicating violations with an exception.

\subparagraph{Dynamic Type Propagation.}
Another optimization could be to use Truffle's approach to 
self-specialization\citep{Wurthinger:2012:SelfOptAST}
and propagate type information to avoid redundant checks.
At the moment, Truffle interpreters typically use self-specialization to 
specialize the AST to avoid boxing of primitive types.
This is done by speculating that some subtree always returns the expected type.
If this is not the case, the return value of the subtree is going to be
propagated via an exception, which is caught and triggers respecialization.
This idea could possibly be used to encode higher-level type information for
return values, too.
This could be used to remove redundant checks in the interpreter
by simply discovering at run time that whole subexpressions conform to the
type annotations.

\subparagraph{Performance Impact of Types}

As seen in \cref{sec:disc-pathological-case},
there are cases where adding types may reduce performance,
even so, in the best case this does not happen (cf. \cref{sec:eval-best-case}).

While the expectation is that adding more types may result in higher potential
for performance issues, in the context of dynamic and adaptive compilation
as used for Moth, this is not necessarily the case.
Since compilers rely on various heuristics, for instance for inlining,
there may be situations where a fully typed program is faster than
a program with fewer types.
Since the checks need to be compiled themselves, they also influence such heuristics.
This means it is possible that
partially typed programs may show worse performance than fully typed ones.

\subsection{Threats to Validity}

%\kjx{this is a first draft, people should fix / check it!}

This work is subject to many of the threats to validity common to
evaluations of experimental language implementations.  Our underlying
implementation may contain undetected bugs that affect the semantics
or performance of the gradual typing checks, affecting construct
validity --- we may not have implemented what we think we have. Given
that, our benchmarking harness run on the same implementation is
subject to the same risks, thus also affecting internal validity ---
we may not be measuring the implementation correctly.  Moth is built
on the Truffle and Graal toolchain, so we expect external validity
there at least --- we expect the results would transfer to other Graal
VMs doing similar AST-based optimizations.  We have less external
validity regarding other kinds of VMs (such as VMs specialized to 
particular languages, or VMs built via meta-tracing rather than partial evaluation). 
Nevertheless, we expect our results should be transferable
as we rely on common techniques.


\subparagraph{Generalizability}
Finally, because we are working in Grace, it is less obvious that our
results generalize to other gradually typed-languages. We have worked
to ensure \eg our benchmarks do not depend on any features of Grace
that are not common in other gradually-typed object-oriented
languages, but as Grace lacks a large corpus of programs the
benchmarks are necessarily artificial, and it is not clear how the
results would transfer to the kinds of programs actually written in
practice. The advantage of Grace (and Moth) for this research is
that their relative simplicity means we have been able to build an
implementation that features competitive performance with significantly less
effort than would be required for larger and more complex languages.
On the other hand, more effort on optimisations could well lead to
even better performance.

Another aspect which limits generalizability is the specific semantics of Grace.
Reticulated Python, Typed Racket, and Gradualtalk have semantics that need
additional runtime support, and thus, we cannot draw any conclusions without
further research.

For languages such as Newspeak, Strongtalk, or TypeScript,
where types do not have run-time semantics,
one could add termination based on type errors to these languages,
or simply avoid termination and report the errors after program completion
as a debugging aid.
For either option, our approach should apply and we would expect similar results.

