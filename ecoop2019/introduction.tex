%!TEX root = ../latex/paper.tex

\section{Introduction}
\label{sec:introduction}

\begin{verse}
  \textit{``It is a truth universally acknowledged, that a dynamic language
    in possession of a good user base, must be in want of a type system.''}
\end{verse}
\vspace*{-6ex}
\begin{flushright}
with apologies to Jane Austen.
\end{flushright}
  


% Despite optional typing becoming increasingly popular in the industry, the systems that offer support for optional typing are not yet fully \emph{alive}. The languages are of course well defined and supported but, in particular, the advantages of the optional typing are out-weighted by their performance costs.
%
% Recently, researchers have explored the potential for type checking to be performed together with execution; rather than as an a stand-alone system that occurs separately at run time. For example, Richards et al. extend the Higgs VM to perform TypeScript's structural checks based on information gathered during method execution.
%
% Tying the type checking system into existing VM behavior can offer faster performance, while also being simple to implement. In particular a compiler can fold the type checking system into the VM's invocation logic and, once compiled, both the number of type checks and their complexity has been reduced. Furthermore, the VM usually has cheap reflective access and therefore the logic of the checking system is simpler to implement than we done on the level of the guest language.

% Context
% - dynamic languges are important
% - much work on performance
Dynamic languages are increasingly prominent in the software industry.
Building on the pioneering work of Self\citep{Self}, 
much work in academia and industry
has gone into making them more efficient\citep{Bolz2013,Bolz:2013:IMT,Wurthinger:2017:PPE,Daloze2016,Clifford:2015:MM,Degenbaev:2016:ITG}.
Just-in-time compilers have, for example, turned JavaScript from a
na{\"\i}vely interpreted language barely suitable for browser scripting, 
into a highly efficient ecosystem, eagerly adopted
%in industry and academia.
by professional programmers for a very wide range of tasks \cite{Pang2018}.

% Types  %%KJX later says: doesn't matter.
% - useful for programmer productivity on large systems
% - document a program's structures
% - optional + gradual
% - allow partial typing
% - are either removed, or have run-time overhead
%%KJX
%% Thanks to these performance gains, dynamic languages are used to build
%% larger and larger systems, which has lead to which leads to typing
%% approaches being adopted to support programmer productivity and
%% document a program's structures.

A key advantage of these dynamic languages is the flexibility offered
by the lack of a static type system. 
From the perspective of many computer scientists, software engineers,
and computational theologists, this advantage has a concomitant
disadvantage in that programs without types are considered more
difficult to read, to understand, and to analyse than programs with
types. Gradual Typing aims to remedy this disadvantage, adding types
to dynamic languages while maintaining their flexibility
\citep{GiladPluggable2004,Siek2006,XXXSiek2015}. 

There is a spectrum of different approaches to gradual typing
\cite{kafka18,bensurvey18icfp}.
At one end\,---\,``pluggable types'' as in Strongtalk \cite{strongtalk} or ``erasure
semantics'' as in 
%% Two important approaches are optional\citep{}
%% and gradual typing\citep{GiladPluggable2004,Siek2006,XXXSiek2015}.
%% These are applied to dynamic languages to reap the benefits of typing, but
%% unfortunately also have limitations.
%With optional or pluggable approaches such as 
%\citeurl{TypeScript,}{TypeScript}{Microsoft}{27 June
%2018}{https://www.typescriptlang.org/}
TypeScript\citep{typeScriptECOOP}\,---\,
%
all types are erased before the execution, limiting the benefit of
types to the statically typed parts of programs, and preventing
programs from depending on type checks at run-time.  In the middle,
``transient'' or ``type-tag'' checks as in Reticulated Python 
% and ``concrete'' checks as in Thorn
offer first-order semantics, checking
whether an object's type constructor or supported methods match
explicit type declarations, and checking only when data flows through
those declarations
\cite{Siek2007,Bloom2009,concrete15,reticPython2014,Greenman2018}.
Reticulated Python also supports an alternative ``monotonic'' semantics
which mutates an object to narrow its concrete type when it is passed
into a more specific type context.
At the other end of the spectrum, behavioural
typechecks as in Typed Racket \cite{typedScheme08,takikawa2012} and Gradualtalk \cite{gradualtalk14} support higher-order semantics, retaining
types until run 
time, performing the checks eagerly, and giving detailed information
about type violations as soon as possible via blame
tracking \cite{blame2009,blameForAll2011}.
%
%\mwh{Does a monotonic semantics fit on this spectrum somewhere?}
%\kjx{Not in the greenman/fellenstein spectrum, but I guess I'd put it
%  after concrete and before behavioural. Does it matter?}
%
Unfortunately, any gradual system with run-time semantics
(i.e.\ everything more complex than erasure) currently
imposes significant run-time performance overhead to provide those semantics
\citep{Takikawa2016,Vitousek2017,Muehlboeck2017,Bauman2017,Richards2017,Stulova2016,Greenman2018}.

The performance cost of run-time checks is problematic in itself,
but also creates perverse incentives. Rather than the ideal of
gradually adding types in the process of hardening a developing
program, the programmer is incentivised to leave the program untyped
or even to \textit{remove} existing types in search of speed.
%
While the Gradual Guarantee~\cite{XXXSiek2015} requires that
removing a type annotation does not affect the result of the
program, the performance profile can be drastically shifted by the
overhead of ill-placed checks.
%
For programs with crucial performance constraints, for new
programmers, and for gradual language designers, juggling this
overhead can lead to increased complexity, suboptimal
software-engineering choices, and code that is harder to maintain,
debug, and analyse.

% our context
% - dynamic language
% - a language for teaching
% - types, execution semantics, consistency
% - no impact on performance

%With the \citeurl{Grace language,}{The Grace Programming
%  Language}{}{}{http://gracelang.org/}


% \kjx{Don't know if we really need this paragraph}
% \mwh{It seems like a vulnerability more than a strength (and the
%   next one more so). I propose the one I've added above to replace it.}
% \kjx{Looks OK now. seems similar to the start of section 3 --- perhpas
%   that's OK?}
% We are working on Grace\citep{graceOnward12}, a dynamic language in
% the tradition of Smalltalk\citep{bluebook}, Self\citep{Self}, and
% JavaScript that is meant for use in
% education\citep{graceSigcse13}.  While Grace is a dynamic language at
% its core, we want to have the option to teach students about types,
% and so Grace supports gradual type annotations which may be checked either
% statically or dynamically to give students feedback on whether their
% type annotations are correct.  We do not want students to remove
% types, however, if they discover that types induce a run-time
% overhead.

% \kjx{or this one}
% Additionally, we are currently maintaining three different
% implementations to support a variety of educational settings
% (web browsers, .NET, and JVM),
% which means a typing approach for Grace ideally requires
% only small changes to keep these implementations as consistent as
% possible.


In this paper, we focus on the centre of the gradual typing
spectrum: the transient, first-order, type-tag checks as used in
Reticulated Python and similar systems. 
Several studies have found
that these type checks have a negative impact on programs'
performance.
Chung, Li, Nardelli and Vitek, for example, found that 
``\textit{The transient approach checks types at uses, so the act of
  adding types to a program introduces more casts and may slow the
  program down (even in fully typed code).}'' and say 
``\textit{"transient semantics\ldots is a worst case scenario\ldots,
  there is a cast at almost every call"}'' \cite{kafka18}.
Greenman and Felleisen find that the slowdown is predictable, as
transient checking ``\textit{imposes a run-time checking overhead that
  is directly proportional to the number of [type annotations] in
  the program"}'' \cite{bensurvey18icfp}, and 
Greenman and Migeed found a ``\textit{clear trend
that adding type annotations adds performance overhead. The increase
is typically linear.}'' \cite{Greenman2018}.  


In contrast, we demonstrate that
transient type checks can be ``almost
free'' via a just-in-time compiler to an
optimizing virtual machine.
We insert 
gradual checks
na\"ively, for each gradual type 
annotation.
% We check types eagerly when control flow reaches them.
Whenever an annotated method is called or
an annotated variable is accessed,
we check types dynamically, and
terminate the program with a type error if the check fails.
Despite this simplistic approach, a just-in-time compiler can
eliminate redundant checks---%
removing almost all of the checking overhead,
resulting in a performance profile aligned with untyped code.

We evaluate our approach by adding transient type checks to Moth,
an implementation of the Grace programming language
built on top of Truffle
and the Graal just-in-time compiler\citep{Wurthinger2013,Wurthinger:2017:PPE}.
Inspired by Richards~\textit{et~al.}~\cite{Richards2017} and
Bauman~\textit{et~al.}~\cite{Bauman2017},
our approach conflates types
with information about the dynamic object structure 
(maps\citep{Self} or object shapes\citep{woss2014object}), 
which allows the just-in-time compiler
to reduce redundancy between checking structure
and checking types; consequently, most of the overhead
that results from type checking is eliminated.

% In this paper, we present a \emph{responds as expected} type-checking system for the Grace programming language. Through \cref{sec:method} we will describe how we extending \SOMns to support Grace and provide support for the simple type checks and, later in \cref{sec:evaluation}, we present two suites of benchmarks that demonstrate our system can offer these checks with little overhead in terms of execution time.

% While we having only taken the first step in rousing Grace's system, our implementation and the results presented are a promising step toward supporting Grace's gradual-structural type system and, ultimately, refuting the notion that gradually typed languages are dead.

The contributions of this paper are:

\begin{itemize}
\item demonstrating that VM optimisations enable
        transient gradual type checks with low performance cost
\item an implementation approach that requires
      only small changes to existing abstract-syntax-tree interpreters
\item an evaluation based on classic benchmarks
      and benchmarks from the literature on gradual typing
\end{itemize}
