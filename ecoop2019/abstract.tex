\begin{abstract}
  It is widely accepted that the additional run-time checks in a
  gradually-typed language must cause a linear slowdown in the program's
  execution.
  %
  This performance impact discourages the use of type annotations
  because adding types can significantly worsen performance, despite the
  benefits of typed code.
  %
  We show that in fact the overhead of transient checks can be near zero
  with appropriate application of standard just-in-time optimization
  techniques.
  %
  With these techniques programmers need not avoid adding types to their
  code and a gradually-typed language can have run-time performance
  comparable to state-of-the-art dynamic language implementations.

%% Removing \kjx{redundant} type tests makes programs in these
%% languages run faster. 
% We add shallow-structural type checking
% to an existing interpreter Grace, 
%that enables developers to check 
%that executions of their program are well-typed. % WTF else would you
%need 'em
%
% Grace is designed primarily for education and, consequently,
% the performance profile of its execution is important
% to enable both teachers and students to explore different concepts.
%
% Following from Grace's design goals,
% our focus is to provide flexible checking
% with minimal overhead to execution.
% We achieve flexibility through our shallow-structural design and,
% for our implementation, 
% we achieve minimal overhead by
% taking advantage of common JIT VM techniques 
% that can remove much of the overhead of dynamic type checking;
% allowing programs with explicit dynamic checks 
% to run at roughly the same speed as programs without those checks.
% Overall, adopting common JIT techniques can make these languages
% more effective in practice.

\end{abstract}
