%!TEX root = paper.tex

\section{Gradual Type Checks in Grace}
\label{sec:method}

% evaluation cases

% high flexibility, no need for global consistency 


% design goals behind Grace / Moth
%As described in \cref{ssec:grace},
%Grace is an educational language and its performance profile
%is important for supporting effective teaching. 
%
%\mwh{It seems like a \textit{remarkably terrible} idea to be proposing
%  this as a contribution, especially when it's conflated and confused
%  with the performance contributions.}
%\kjx{we wont}
%
% motivation = give students low-overhead method to check program
The static core of Grace's type system is well described
elsewhere\citep{TimJonesThesis};%
~here we explain how
these types can be understood 
dynamically, from the Grace programmer's point of view.
%\kjx{code for: no formalism, sorry.}
% Following from the design goals behind Grace,
Our motivation for this work
is to provide a flexible system 
to check consistency between an execution of a program
and its type annotations,
without significantly impacting run-time performance.
A secondary goal is to have a design that can be implemented with
only a small set of changes to facilitate integration in existing systems.
%
% gradual typing drawback = slow
%\chg{Gradual typing systems are generally good candidates to achieve
%this goal.}{Gradual typing systems are good candidates,because they
%provide the gradual guarantee and the mechanisms for blame would be desirable.}
% Existing gradual typing systems 
% that address the goal of flexibility currently
% Unfortunately, they do not yet have the desired performance properties\citep{Takikawa2016,Vitousek2017,Muehlboeck2017,Bauman2017,Richards2017,Greenman2018}.
% Specifically, they could encourage students to remove types to improve performance.
%


These goals are shared with
much of the other work on gradual type systems,
but our context leads to some different choices.
First,
while checking Grace's type annotations statically may be optional,
checking them dynamically should not be:%
~any value that flows into a variable, argument, or result
annotated with a type must conform to that type annotation.
This means our focus is not to devise a sound typing approach,
but rather an approach that ensures that the observed execution matches
the one the Grace programmer expects when considering a program's type annotations.
Second, 
adding type annotations should not degrade a program's performance,
or rather, programmers should not be encouraged to
improve performance by removing type annotations.
And third, 
we allow the programmer to execute a program even when not statically type-correct.\footnote{Grace's static type checking is optional, and so an
implementation cannot depend on the correctness or mutual
compatibility of a program's type
annotations.}
Allowing such execution is useful to students,
where they can see concrete examples of dynamic type errors.
%However, reported errors should conform to the basic expectations one
%may derive from the type annotations
%and a strict interpretation of their shallow semantics.

Unfortunately, existing gradual type
implementations do not meet these goals, particularly regarding
performance; hence the ongoing debate about whether gradual typing is
alive, dead, or some state
in between\citep{Takikawa2016,Vitousek2017,Muehlboeck2017,Bauman2017,Richards2017,Greenman2018}.
% Specifically, they could encourage students to remove types to improve performance.
% \mwh{This paragraph seems confused to me. They're good, but also they're bad. What is a ``gradual typing system'' here, and how does it relate to ``gradual typing'' and ``type systems'' as separate concepts?}
% \sm{the above change makes the desirable things about gradual typing more concrete. wrt ``gradual typing'' and ``type systems''. one is a general idea, and a concrete realization is a ``system''. I don't know whether this is how typing people talk. but this is how I see the distinction.
% we don't want to criticize the abstract idea, but concrete systems.}
% \kjx{I've rewritten this bit, particularly the last paragraph.}

%% TODO: (SM) need to look at this for revision:

% \kjx{I'm not sure if the goals should go here, or in the intro?}
% \kjx{now we have two different lists of goals here}
% \kjx{I'm more than half temped to just delete all of 3.0 --- but I've left
% it for now.}

% % KJX likes this bit, wrote it below, doesn't seem to belong there
% % 
% % Part of the philosophy of Grace is that the language should not force
% % students to annotate programs with types until they are ready, so that
% % teachers can choose whether to introduce types, early, late, or even
% % not a all.  Assuming 



% % summary of goals
% Instead, the design behind our type system is focused
% on maximizing performance and flexibility while
% preserving the ability for programmers to check the execution
% of their programs.

% % how we address flexible
% To address the goal of flexibility we propose a system that is optional,
% which enables the programmer to benefit from adding checks
% to be performed at run time,
% without the burden of needing to fully type each program.

% how we address min overhead (avoid blame)
% that enables the executed behavior of the program to be checked against
% the documented types without the
% (which leads to significant overhead in previous work).


\kjx{kill the following heading?}
\subsection{Design}


%% TODO: (SM) need to look at this for revision:
% \kjx{I worked though this mostly doing terminology,
% and perhaps moving from claims to descriptions.
% Does this make sense? is it any better?}

% JN killed
% To achieve the consistency checking with minimal run-time overhead,
% we propose a system with shallow structural type checks,
% which provide useful feedback without requiring a blame mechanism
% as in typical gradually typed approaches.

% Message sending
% As we described in \cref{ssec:grace}, Grace is message-sending-based language.

%

% \kjx{adjusted the order through here}

In Grace\citep{graceOnward12},
an object implements a type whenever it
implements all the methods required by that type,
rather than requiring classes or objects to declare types explicitly.
Methods match when they have the same name and arity.
A type thus expresses the requests an object can respond to,
for example whether a particular accessor is available,
rather than a nominal location in an explicit inheritance hierarchy.

Our gradual type checks for Grace are designed to be simple,
straightforward, and easy to understand:
\begin{itemize}
  \item types are sets of methods (first-order interfaces)
  \item all gradual type annotations are checked at run time
  \item failing run-time type checks terminate execution
\end{itemize}
%
\kjx{suggest adding the next sentence - tho' I'm not sure how the Grace's
  design stuff really works overall}
%
In the spectrum of gradual typing, these semantics are essentially the
transient semantics of Reticulated Python
\cite{reticPython2014,Greenman2018} ---
with the minor differences that Grace's semantics
treat field access as method calls, and we check both reads and
writes of variables, along with return values on the receiver side
(see \S\ref{reticRW}).
%% \kjx{can Michael check this?}
%% \mwh{I think this is correct and a difference, but I'm not sure
%%   exactly what Moth does. I think it should check on both caller and
%%   receiver (and they will be collapsed if possible). This phrasing I
%%   think is not inaccurate but perhaps imprecise.
%% }

% Together these properties enable a flexible system
% with minimal overhead
% that developers can use to verify that programs
% execute with values that have the capabilities
% documented through their annotations
% while avoiding the burden of fully typing their program.

\begin{lstlisting}[caption={The start of a simple program for tracking vehicle information.},float,label=lst:car-reg,escapechar=|,columns=flexible,float,floatplacement=H]
def car = object {|\label{ex:object}|
    var registration is public := "JO3553"
}

method printRegistration(v) {|\label{ex:method}|
    print "Registration: {v.registration}"
}
\end{lstlisting}



Finally,
we now illustrate how the gradual type checks work in practice
in the context of a simple program to record information about vehicles.
Suppose the programmer starts developing this vehicle
application by defining an object intended to represent a car
(\cref{lst:car-reg}, \cref{ex:object}) and writes a method that, given
the car object, prints out its registration number (\cref{ex:method}).
%
Next, the programmer adds a check to ensure any object passed to the
\code{print\-Reg\-is\-tra\-tion} method will respond to the
\code{registration} request; 
they define the structural type \code{Vehicle}\citep{theCleanVehicle}
naming just that method (\cref{ex:vehicle}, \cref{ex:adding-type:vehicle}), 
and annotate the \code{printRegistration} method's
argument with that type (\cref{ex:vehicle}, \cref{ex:adding-type}).
The annotation ensures that a type error will be thrown if an object,
passed to the \code{printRegistration} method,
cannot respond to the \code{registration} message.
Furthermore, as type errors constituent termination, 
a crash somewhere in the middle of the
implementation of the \code{print} method
will now be avoided.

\begin{lstlisting}[label={ex:vehicle},caption={Adding a type annotation to a method parameter.},escapechar=|,columns=flexible,float,floatplacement=H]
type Vehicle = interface { |\label{ex:adding-type:vehicle}|
    registration    
}

method printRegistration(v: Vehicle) { |\label{ex:adding-type}|
    print "Registration: {v.registration}"
}
\end{lstlisting}


% \sm{you already said the following}
% By adding the type annotation to the program,
% the student can now be sure the \code{getReg} method
% will only be invoked when its arguments can respond to this message.

%\paragraph{Flexibility.}
%\paragraph{Semnatics}

% \kjx{isn't most of this covered by the earlier para above now? -- so
%   can we delete this one? please!?}
% While Grace's static type system supports full static type
% checking\citep{graceOnward12}, Grace's specification requires dynamic
% type tests to be \emph{shallow}, that is, they check only for the
% presence and arity of methods in an object, rather than also checking
% conformance of argument and result types.  This is to ensure that the
% presence or absence of type annotations does not affect the execution
% of a program, for the reason originally outlined by
% Boyland~\cite{Boyland2014}, thus maintaining a version of the gradual
% guarantee.  The resulting semantics are \del{more-or-less}\sm{we really need to say something stronger here or drop this} equivalent to
% type-tag soundness\citep{reticPython2014,Vitousek2017,Greenman2018}.
% \kjx{now saying ``equivalent'', punting fine details to the related work section - but
%   I suggest we kill this para}.

%% SM: someone should really take a really close look at these and identify the
%%     differences. And I don't want to do that.
% ---the difference being that
% where type-tag soundness supports shallow \emph{nominal} type checks,
% we support shallow \emph{structural} type checks.
%%KJX: my comment above is plan WRONG - Retic is structural too.
%% main differences are method returns, object assignments.

% The notion of our type system being shallow means 
% that members of a type are untyped.
% In particular, our design does not have information on parameter types
% and return types for types' members.
% This design implies that blame tracking is not needed,\mwh{Um.}
% because the types are less detailed than in other systems.\mwh{I would hope that a competent reviewer would't fall for this\ldots}\sm{ok, what can we do about this? what's the precise issue here?
% The idea is that we do not need blame, because we do not have the type casts
% from blame-supported gradual systems.
% The difference is, I think that there is nothing we can not check immediately,
% and error as soon as there is a type.
% I believe the examples are usually higher-order functions/blocks.
% We only check the arity. So, there is no type cast wrapping necessary,
% when we pass it into a method that would expect another type of block arguments.
% (because passing in should not error, the error should only happen when
% the block is used in these systems, right?)}
% One reason for this design is that it avoids the overhead incurred
% by tracking type assumptions for precise blame attribution.



% Another aspect of this design is
% that our shallow approach to types
% allows for more flexible use 
% without requiring type parameters, \ie, generics.
In \cref{ex:complex}, 
the programmer continues development and creates two car objects 
(\cref{ex:personal-car,ex:government-car}),
that conform to an expanded \code{Vehicle} type (\cref{ex:new-vehicle}).
Note that each version of the \code{registerTo} method
declares a different type for its parameter
(\cref{ex:personal-car:registerTo,ex:government-car:registerTo}).
% Depending on its semantics,
% a less-shallow type checking system could throw an error
% due to the inconsistency.
% In contrast
% our approach allows the student to execute the program 
% despite its inconsistency, 
% while preserving the guarantee that any value found to be
% inconsistent with the annotation will result in
% a termination by a typing error.
When the programmer executes this program,
both \code{personal\-Car} and \code{governmentCar} can be assigned to
a variable declared as \code{Vehicle} because checking that assignment considers only that the vehicle has
a \code{registerTo} method, but not the required argument type of that
method.
%
At \cref{ex:invoke-register-to} the developer
attempts to register a government car to a person:%
~only when the method is \textit{invoked} (\cref{ex:government-car:registerTo})
will the gradual type test on the argument fail
(the object that is passed in is not a \code{Department} because it lacks a
\code{code} method).
%The ability to execute statically invalid programs,
%is valuable because 
%it enables developers to test
%partially implemented programs and also
%lets students see concrete examples of dynamic type errors.
%\mwh{Should this/somewhere explain \textit{why} that is? It sounds a bit like just describing a flaw.} \rr{Added a further sentence, do you think its suitable?}

% \sm{don't say anything about stuff we don't do. this is not really necessary,
% and I don't think this is necessarily correct, depending on specific type systems}
% A full structural type-checking system would alert the student to
% the inconsistency described above statically,
% perhaps displaying an error in their idea.
% While the alert is correct in that their is an inconsistency,
% such a system is less flexible.

% \sm{I would just say: possible inconsistency between elements can be approached
% step wise, when necessary as the sophistication and completeness of the program improves}
% Our system offers a higher level of flexibility in that the student
% may still execute the program,
% which may indeed be only partially developed and remain globally inconsistent.
% The student may be satisfied that a particular test passes successfully, and
% is then free to address the inconsistency in later development.

\begin{lstlisting}[caption={A program in development with inconsistent
    types.},escapechar=|,label={ex:complex},float,floatplacement=htb,columns=flexible,float,floatplacement=H]
type Vehicle = interface { |\label{ex:new-vehicle}|
    registration
    registerTo(_)
}

type Person = interface { name }
type Department = interface { code }

var personalCar : Vehicle := |\label{ex:personal-car}|
  object {
    var registration is public := "DLS018"
    method registerTo(p: Person) {|\label{ex:personal-car:registerTo}|
      print "{p.name} registers {self}"
    } 
  }

var governmentCar : Vehicle := |\label{ex:government-car}|
  object {
    var registration is public := "FKD218"
    method registerTo(d: Department) { |\label{ex:government-car:registerTo}|
      print "some department {self}"
    }
  }

governmentCar.registerTo( |\label{ex:invoke-register-to}|
  object {
    var name is public := "Richard"
  }
)
\end{lstlisting}


%\paragraph{Termination by Type Error.}
% \label{sec:term-type-error}

% When executing a program without types, there are three possible outcomes.
% Either the program (1) terminates successfully,
% (2) terminates with an exception, or 
% (3) the execution diverges, \ie, it does not terminate.
% Using our approach, a
% fourth outcome is possible: termination with a type error.
% Our implementation checks  every type annotation
% on the values of arguments before invoking a method, 
% on the value returned by a method when it returns, and
% before any assignment to and read from a variable 
% (either local to a method or belonging to an object). 

% The checks are performed eagerly%
% ---as soon as they are encountered during execution---%
% and cause the execution to terminate with a typing error 
% when a value fails to implement its expected type.

% % kjx deleted this summary because it doesn't add anything.
% % 
% % % Summarize the design section
% % \paragraph{Summary}
% % Our type checking approach enables developers to express
% % the capabilities of objects throughout different components of 
% % their programs.
% % Our representation of types is shallow,
% % in that a type expresses only the set of members an object
% % should offer, while excluding any further typing information. 
% % The shallow design enables a flexible use of structural types without
% % requiring type parameters.
% % We abort program execution as soon as
% % a type annotation is inconsistent with a concrete value. 
% % With these design choices, our system offers a
% % mechanism to check for well-typed executions%
% % ---rather than well-typed programs---%
% % without negatively affecting the performance profile.
% % % This both lowers overhead for the developer and,
% % % helps us to avoid the overhead of tracking blame
% % % as seen among previous works.

\subsection{Implementation} 
\label{ssec:implementation} 

%This section gives an overview of a possible implementation
%based on an abstract-syntax-tree (AST) interpreter.

We have implemented our gradual type checks 
by extending Moth, 
an abstract-syntax-tree (AST) interpreter for
Grace (\cref{ssec:moth}).
%
% We developed our implementation as an extension to Moth.
% As described earlier in \cref{ssec:moth},
% Moth is an AST-based interpreter on top of the Graal VM.
%
% It is optimizes itself based on for instance run-time type information.
%
%
%Based on \cref{sec:term-type-error},
Our approach needs to check types of values at run-time:

\begin{itemize}
\item the values of arguments are checked after a method is requested, 
      but before the body of the method is executed,
\item the value returned by a method is checked after its body is executed, and
\item the values of variables are checked
      whenever written or read by user code.
\end{itemize}

% \kjxdone{why both reading and writing?  answer - doesn't matter?}

One of the goals for our approach to gradual typing was to keep
the necessary changes to an existing implementation small,
while enabling optimization in highly efficient language runtimes.
%
In an AST interpreter, we can implement this approach by attaching the
checks to the relevant AST nodes: the expected types for the argument
and return values can be included with the node for requesting a
method, and the expected type for a variable can be attached to the
nodes for reading from and writing to that variable.  In practice, we
encapsulate the logic of the check within a new class of AST
nodes, specially to support gradual type checking.  Moth's front end was adapted to parse and record type
annotations and attach instances of this checking node as children of the
existing method, variable read, and variable write nodes.


% The check node is detailed in \cref{ssec:optimization} to discuss
% relevant optimizations.

%

The check node uses the internal representation of a Grace type
(cf. \cref{ex:type}, \cref{ex:type:check}) to test whether an observed
object conforms to that type. 
% These \code{Type} objects are created by
% Grace \code{interface} expressions, and also help
% to support Grace's pattern matching facilities \cite{gracePatternsDLS12}.
An object satisfies a type if all members required by the type are provided
by that object (\cref{ex:type:satisfied}).


\begin{lstlisting}[label={ex:type},escapechar=|,caption={Sketch of a \code{Type} in our system and its \code{check()} semantics.},float,floatplacement=htb,columns=flexible,float,floatplacement=H]
class Type:
  def init(members):
    self._members = members

  def is_satisfied_by(other: Type): |\label{ex:type:satisfied}|
    for m in self._members:
      if m not in other._members:
        return False
    return True

  def check(obj: Object):
    t = get_type(obj)
    return self.is_satisfied_by(t) |\label{ex:type:check}|
\end{lstlisting}


\subsection{Optimization}
\label{ssec:optimization}

There are two aspects to our implementation that are critical for a minimal-overhead solution:

\begin{itemize}
  \item specialized executions of the type checking node, along with guards to protect these specialized versions, and
  \item a matrix to cache sub-typing relationships to eliminate
    redundant exhaustive subtype tests.
\end{itemize}
 
%Here we discuss each of the aspects in more detail.

\begin{lstlisting}[label={ex:typenode},escapechar=|,caption={An illustration of the type checking node that support type checking},float,floatplacement=htbp,columns=flexible,morekeywords={global}]
global record: Matrix

class TypeCheckNode(Node):

  expected: Type

  @Spec(static_guard=expected.check(obj))
  def check(obj: Number): |\label{ex:typenode:number}|
    pass

  @Spec(static_guard=expected.check(obj))
  def check(obj: String): |\label{ex:typenode:string}|
    pass

  ...

  @Spec(
      guard=obj.shape==cached_shape,
      static_guard=expected.check(obj))
  def check(obj: Object, @Cached(obj.shape) cached_shape: Shape): |\label{ex:typenode:object}|
    pass
  
  @Fallback
  def check_generic(obj: Any): |\label{ex:typenode:generic}|
    T = get_type(obj)
    
    if record[T, expected] is unknown: |\label{ex:typenode:matrix}|
      record[T, expected] =
          T.is_subtype_of(expected) |\label{ex:typenode:reuse}|

    if not record[T, expected]:
      raise TypeError(
          "{obj} doesn't implement {expected}")
\end{lstlisting}

The first performance-critical aspect to our implementation
is the optimization of the type checking node.
We rely on Truffle and its TruffleDSL\citep{humer2014domainspecific}.
This means we provide a number of special cases,
which are selected during execution based on the observed concrete 
kinds of objects.
A sketch of our type checking node using a pseudo-code version of the DSL
is given in \cref{ex:typenode}.
A simple optimization is for well known types such as
numbers (\cref{ex:typenode:number}) or strings (\cref{ex:typenode:string}).
The methods annotated with \code{@Spec} (shorthand for \code{@Specialization})
correspond to possible states in a state machine that is generated by the
TruffleDSL.
Thus, if a check node observes a number or a string,
it will check on the first execution only that the expected type,
\ie, the one defined by some type annotation,
is satisfied by the object by using a \code{static\_guard}.
If this is the case, the DSL will activate this state.
For just-in-time compilation, only the activated states and their normal guards are considered.
A \code{static\_guard} is not included in the optimized code.
If a check fails, or no specialization matches, the fallback,
\ie, \code{check\_generic} is selected (\cref{ex:typenode:generic}),
which may raise a type error.

For generic objects, we rely on the specialization on \cref{ex:typenode:object},
which checks that the object satisfies the expected type.
If that is the case, it reads the shape of the object (cf. \cref{ssec:moth}) at specialization time,
and caches it for later comparisons.
Thus, during normal execution,
we only need to read the shape of the object and then compare it to the cached shape
with a simple reference comparison.
If the shapes are the same, we can assume the type check passed successfully.
Note that shapes are not equivalent to types,
however, shapes imply the set of members of an object, and thus,
do imply whether an object fulfills one of our structural types.

The other performance-critical aspect to our implementation
is the use of a matrix to cache sub-typing relationships.
The matrix compares types against types,
featuring all known types along the columns and the same types again along the rows.
A cell in the table corresponds to a sub-typing relationship:
does the type corresponding to the row implement
the type corresponding to the column?
All cells in the matrix begin as unknown and, as
encountered in checks during execution, we populate the table.
If a particular relationship has been computed before
we can skip the check and instead recall the previously-computed value
(\cref{ex:typenode:reuse}).
Using this table we are able to eliminate the redundancy of evaluating
the same type to type relationships across different checks in the program. To reduce redundancy further we also unify types in a similar way to Java's string interning; 
during the construction of a type we first check to see if the same
set of members is expressed by a previously-created type and, if so,
we avoid creating the new instance and provide the existing one instead.


Together the self-specializing type check node and the cache matrix 
ensure that our implementation eliminates redundancy, and
consequently, we are able to minimize the run-time overhead of our system. 

%%let's nor worry about that now.  another whole paper at least
%% \smtodo{Yeah, then the insight here is that we can’t communicate
%%   “purity”. And generally, I think the insight with many kinds of
%%   language implementations is that these methods we may want to be
%%   “visibly pure”/idempotent rarely are provably so.
%%   \ldots Instead, the problem is avoided by being explicit about when
%%   values are valid.} 


% The Graal compiler 
% Moth needs to query the shape of arguments to evaluate them.
% Moth then needs to query the shape again to run the specializations' guards.
% Once profiled Graal can remove this redundancy by eliminating
% reusing the first query.
% During this compilation Graal also employs other optimizations 
% -- sub-expression elimination, inlining, propagation and folding --\tabularnewline
% and, as we demonstrate next in \cref{sec:evaluation},
% produces a compiled version of the method with minimal overhead incurred from
% the type check.
