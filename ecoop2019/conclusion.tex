%!TEX root = paper.tex

\section{Conclusion}
\label{sec:conclusion}

%% Context
As gradually typed languages become more common,
and both static and dynamically typed languages are
extended with gradual features,
efficient techniques for gradual type checking become more important.
%
%% goals
%% approach to support types in a dynamic language
%% - without significant overhead
%% - simple to implement
%
%
%
%
% This work proposes a new design for checking shallow structural
%  types at run time.
%  The main goal of the design was to enable an implementation that
%  does not introduce significant run time overhead.
%  A secondary goal was to be able to implement this approach for
%  existing language implementations with only a small amount of work.
%
In this paper, we have demonstrated that optimizing virtual machines enable
transient gradual type checks with relatively little
overhead, and with only small modifications to an AST interpreter.
We evaluated this approach with Moth, an implementation of the Grace language
on top of Truffle and Graal.
%
% We propose a new design for checking shallow structural
%  types at run time.
%  The main goal of the design was to enable an implementation that
%  does not introduce significant run time overhead.
%  A secondary goal was to be able to implement this approach for
%  existing language implementations with only a small amount of work.
%
%
%% proposed technique
%% - shallow structural types
%The proposed approach to shallow types considers only the names of
%type members. 
% Thus, it does not consider types on these members.
% With this shallow design, we enable flexibility in the sense that
% globally inconsistent programs can still be executed and
% checked for consistency with their annotations as far as
% that particular execution goes%
% --- an attribute that supports an incremental style of development
% that may be more common among students learning programming---%.
% Furthermore, it enables precise feedback on type violations
% without requiring blame tracking,
% as it is usually needed for approaches to gradual typing
% at the cost of run-time performance.

%% - type checks, optimization based on object shapes
In our implementation, types are structural and shallow: a type
specifies only the names of members provided by objects, and not
the types of their arguments and results.
These types are checked on access to variables,
when assigning to method parameters, and also on return values.
The information on types is encoded as part of an object's shape,
which means that shape checks already performed in an optimizing dynamic
language implementation can be used to check types, too. 
Being able to tie checks to the shapes in this way is critical for 
reducing the overhead of dynamic checking.
%
%% results
%% + context (baseline)

Using the \AWFY benchmarks as well as a collection of benchmarks from the
gradual typing literature, we find that our approach to dynamic type checking
introduces an overhead of 
\OverheadTypingGMeanP (min. \OverheadTypingMinP, max. \OverheadTypingMaxP)
on peak performance.
In addition to the results from further microbenchmarks,
we take this as a strong indication that transient gradual types do not
need to imply a linear overhead compared to untyped programs.
However, we also see that interpreter and startup performance is indeed
reduced by additional type annotations.

% \begin{note}
%   Add this:
%   1. peak performance with a “replace type-check
%    by shape-approach” (as known from literature and as we do) does not suffer 
%    from this problem in dynamic languages that need shape checks anyway.
%   Caveat: there are checks that don’t correspond to shape checks: 
%    assignments as in the List benchmark.
  
%   2. Interpreter/warmup performance: there we still pay these costs.
%    So, there can still be a non-trivial performance impact of additional
%    type annotations until we reach optimally compiled code.
  
%   possibly mention this point again in the conclusion
%   \end{note}

Since Moth reaches the performance of a
highly optimized JavaScript VM such as V8,
we believe that these results are a good indication
for the low peak-performance overhead of our approach.

%% relevance
%%   where to go from here?

%% future work
In specific cases, the overhead is still significant and requires further
research to be practical. Thus, future research should investigate how the
number of gradual type checks can be reduced without causing
the type feedback to become too imprecise to be useful.
One approach might increase the necessary changes to a language implementation,
but avoid checking every variable read.
Another approach might further leverage Truffle's self-specialization
to propagate type requirements and avoid unnecessary checks.
%%  - investigate optimizations
%%  - might enable a full structural type system, \ie, not shallow
%%  - verify results of Richards2017 in context of highly optimizing VM

Finally, we hope to apply our approach to other parts of the spectrum
of gradual typing, eventually reaching 
full structural types with
blame that support the gradual guarantee.  
This should let us verify that
Richards~\textit{et~al.}~\cite{Richards2017}'s results generalize to highly optimizing virtual
machines, or alternatively, show that other optimizations for precise
blame need to be investigated.

