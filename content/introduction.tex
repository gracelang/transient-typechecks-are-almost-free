%!TEX root = ../latex/paper.tex

\section{Introduction}
\label{sec:introduction}

% Despite optional typing becoming increasingly popular in the industry, the systems that offer support for optional typing are not yet fully \emph{alive}. The languages are of course well defined and supported but, in particular, the advantages of the optional typing are out-weighted by their performance costs.
%
% Recently, researchers have explored the potential for type checking to be performed together with execution; rather than as an a stand-alone system that occurs separately at run time. For example, Richards et al. extend the Higgs VM to perform TypeScript's structural checks based on information gathered during method execution.
%
% Tying the type checking system into existing VM behavior can offer faster performance, while also being simple to implement. In particular a compiler can fold the type checking system into the VM's invocation logic and, once compiled, both the number of type checks and their complexity has been reduced. Furthermore, the VM usually has cheap reflective access and therefore the logic of the checking system is simpler to implement than we done on the level of the guest language.

% Context
% - dynamic languges are important
% - much work on performance
Dynamic languages are increasingly prominent in the software industry.
Building on the pioneering work of Self\citep{Self}, 
much work in academia and industry
has gone into making them more efficient\citep{Bolz2013,Bolz:2013:IMT,Wurthinger:2017:PPE,Daloze2016,Clifford:2015:MM,Degenbaev:2016:ITG}.
Just-in-time compilers have taken JavaScript, for example, from a
na{\"\i}vely interpreted language barely suitable for browser scripting, 
to a highly efficient ecosystem that is sweeping across industry
and academia\citep{githut2018}.

% Types
% - useful for programmer productivity on large systems
% - document a program's structures
% - optional + gradual
% - allow partial typing
% - are either removed, or have run-time overhead

With these performance gains,
dynamic languages are used to build larger and larger systems,
which leads to typing approaches being adopted
to support programmer productivity and document a program's structures.
Two important approaches are optional\citep{GiladPluggable2004}
and gradual typing\citep{Siek2006,Siek2015}.
These are applied to dynamic languages to reap the benefits of typing, but
unfortunately also have limitations.
With optional or pluggable approaches such as 
%\citeurl{TypeScript,}{TypeScript}{Microsoft}{27 June
%2018}{https://www.typescriptlang.org/}
TypeScript\citep{typeScriptECOOP,GiladPluggable2004}
%
types are erased before the execution,
limiting the benefit of types to the statically typed parts of programs.
In contrast, gradual type systems retain types until run time,
performing the checks dynamically, and
can give detailed information about type violations via blame 
tracking\citep{Siek2015,blame2009}.
Unfortunately, these gradual systems currently impose significant
run-time overheads
\citep{Takikawa2016,Vitousek2017,Muehlboeck2017,Bauman2017,Richards2017,Stulova2016,Greenman2018}.

% our context
% - dynamic language
% - a language for teaching
% - types, execution semantics, consistency
% - no impact on performance

%With the \citeurl{Grace language,}{The Grace Programming
%  Language}{}{}{http://gracelang.org/}
We are working on Grace\citep{graceOnward12}, a dynamic language in
the tradition of Smalltalk\citep{bluebook}, Self\citep{Self}, and
JavaScript that is meant for use in
education\citep{graceSigcse13}.  While Grace is a dynamic language at
its core, we want to have the option to teach students about types,
and so Grace supports type annotations which may be checked either
statically or dynamically to give students feedback on whether their
type annotations are correct.  We do not want students to remove
types, however, if they discover that types induce a run-time
overhead.

Additionally, we are currently maintaining three different
implementations to support a variety of educational settings
(web browsers, .NET, and JVM),
which means a typing approach for Grace ideally requires
only small changes to keep these implementations as consistent as
possible.

In this paper we illustrate that using an optimizing virtual machine allows dynamic
checks of shallow structural types with low overhead and relatively low
implementation effort. These checks are inserted na\"ively based on local
annotations and checked eagerly when control flow reaches them: 
whenever an annotated method is called or
an annotated variable is accessed
we check types dynamically and
terminate the program with a type error if the check fails.
Despite this simplistic approach, a just-in-time compiler can
elminate the redundant checks---%
removing almost all of the checking overhead,
resulting in
a performance profile aligned with untyped code.

We evaluate this approach with Moth,
a Grace implementation on top of Truffle
and the Graal just-in-time compiler\citep{Wurthinger2013,Wurthinger:2017:PPE}.
Inspired by \citet{Richards2017} and \citet{Bauman2017},
our implementation conflates types
with information about the dynamic object structure 
(maps\citep{Self} or object shapes\citep{woss2014object}), 
which allows the just-in-time compiler
to reduce redundancy between checking structure
and checking types; and consequently, most of the overhead 
that results from type checking is eliminated.

% In this paper, we present a \emph{responds as expected} type-checking system for the Grace programming language. Through \cref{sec:method} we will describe how we extending \SOMns to support Grace and provide support for the simple type checks and, later in \cref{sec:evaluation}, we present two suites of benchmarks that demonstrate our system can offer these checks with little overhead in terms of execution time.

% While we having only taken the first step in rousing Grace's system, our implementation and the results presented are a promising step toward supporting Grace's gradual-structural type system and, ultimately, refuting the notion that gradually typed languages are dead.

The contributions of this paper are:

\begin{itemize}
\item demonstrating that VM optimisations enable
        dynamic checks of shallow structural types with low
        performance cost
\item an implementation approach that requires
      only small changes to existing abstract-syntax-tree interpreters
\item an evaluation based on classic benchmarks
      and benchmarks from the literature on gradual typing
\end{itemize}
