%!TEX root = ../latex/paper.tex

\section{Introduction}
\label{sec:introduction}

Despite optional typing becoming increasingly popular in the industry, the systems that offer support for optional typing are not yet fully \emph{alive}. The languages are of course well defined and supported but, in particular, the advantages of the optional typing are out-weighted by their performance costs. 

Recently, researchers have explored the potential for type checking to be performed together with execution; rather than as an a stand-alone system that occurs separately at run time. For example, Richards et al. extend the Higgs VM to perform TypeScript's structural checks based on information gathered during method execution. 

Tying the type checking system into existing VM behavior can offer faster performance, while also being simple to implement. In particular a compiler can fold the type checking system into the VM's invocation logic and, once compiled, both the number of type checks and their complexity has been reduced. Furthermore, the VM usually has cheap reflective access and therefore the logic of the checking system is simpler to implement than we done on the level of the guest language.

In this paper, we present a \emph{responds as expected} type-checking system for the Grace programming language. Through \cref{sec:method} we will describe how we extending \SOMns to support Grace and provide support for the simple type checks and, later in \cref{sec:evaluation}, we present two suites of benchmarks that demonstrate our system can offer these checks with little overhead in terms of execution time.

While we having only taken the first step in rousing Grace's system, our implementation and the results presented are a promising step toward supporting Grace's gradual-structural type system and, ultimately, refuting the notion that gradually typed languages are dead.

\sm{pick up on James argument about teaching here:
we don't want people to remove types for performance}
\sm{other points discussed:
we do want to make sure that the executed parts of a program are self-consistent, so, we need to check the type annotations}
\sm{needs to be implementable without much changes, because there are multiple Grace
implementations, and maintaining them is huge effort}

in context of highly optimize run time system

\begin{itemize}
\item helps developer be more productive
\item check executed parts of program are consistent with its annotations
\item small number of changes to implement
\item small overhead
\end{itemize}