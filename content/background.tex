%!TEX root = ../latex/paper.tex

\section{Background}
\label{sec:background}

\subsection{Stronger Typing for Dynamically Typed Languages}

\kjx{No idea why we need this subsection}

\begin{cnote}
- the dynamic typing discussion as in the outline
\end{cnote}

\subsection{The Grace Programming Language}

Grace is an object-oriented, imperative, educational programming
language we are desinging, with a focus on introductory programming
courses, but also for more advanced study and research
\cite{graceAbsence,seekinGraceSIGCSE}.  While Grace's syntax draws
from the so-called ``curly bracket'' tradition of C, Java, and
JavaScript (with a side order of Pascal) the structure of the language
is in many ways closer to Smalltalk (thus Self and Ruby): method names
have multiple parts, blocks (lambdas) are used for control flow, and
returns within lambdas are ``non-local'', returning to the method
activation in which the block is instantiated \cite{bluebook}.  In
other ways, Grace is closer to Javascript than Smalltalk: Grace
objects can be created from object literals, rather than by
instantiating classes \cite{Black2007-emeraldHOPL,JonesECOOP2016} and
objects and classes can be deeply nested within each other
\cite{betabook}.  Finally, Grace's declarations and methods' arguments
and results can be annotated with types.  These types can be checked
either statically or dynamically. This means the type system is
optional or ``pluggable'' \cite{GiladPliggable2004} (removing explicit
type annotations should not affect the semantics of a correct program
\cite{gradualGuaratee}) and gradual (the type system includes a
distinguished ``Unknown'' type that reverts to fully
dynamic type checking).

As an educational language \cite{panel}, absolute performance of an
implementation is less important than the performance profile --- the
way language features affect performance.  increasing absolute
performance by several orders of magnitude could let students run
larger examples --- analysing billions rather than millions of data
points, wayfinding within a city rather than a village, raytracing
higher resolution images a little quicker.  On the other hand, issues
with a language's performance profile could mean the students will
``learn the wrong things''.  If e.g.\ a languages' built-in cons lists
were faster than arrays or hash-tables, students cannot learn the
performance benefits of more complex data structures. In the case of
Grace, all extant implementations have the unfortunate property that
adding type declarations to a program makes that program run slower
--- teaching students that \emph{removing} type declarations is an
effective optimisation technique.  Furthermore, this property is shared
by several other optionally typed languages including Groovy,
Typescript and Dart \cite{find-who-says-this-and-cite-them}.


\subsection{Moth: Grace on Graal and Truffle}
\label{ssec:moth}

Implementing dynamic languages as state-of-the-art virtual machines
can require enormous engineering efforts.
Meta-compilation approaches\citep{Marr:2015:MTPE}
such as RPython\citep{Bolz:2009:TMP,Bolz:2013:IMT}
and GraalVM\citep{Wurthinger2013,Wurthinger:2017:PPE}
reduce the necessary work dramatically,
because they allow language implementers to leverage existing VMs
and their support for just-in-time compilation and garbage collection.

To leverage this infrastructure, we developed an interpreter for Grace,
called Moth\citep{Roberts2017}, by adapting
\citeurl{\SOMns.}{SOMns: A Newspeak for Concurrency Research}{Stefan Marr}{}{https://github.com/smarr/SOMns/}
\SOMns is a Newspeak implementation\citep{Bracha:10:NS} on top of the Truffle framework and the Graal just-in-time compiler,
which are part of the GraalVM project.
Since Newspeak and Grace are related languages,
\SOMns provides a good foundation for a new Grace implementation,
allowing us to reach the performance of V8,
Google's JavaScript implementation
(cf. \cref{sec:baseline-perf} and \citet{Marr2016})
with only moderate effort.
\SOMns was changed to parse Grace code.
\SOMns' self-optimizing abstract-syntax-tree nodes were only slightly adapted to conform to Grace's semantics.
As a result, Moth is \ugh{mostly compliant}\sm{do we need to make this
  more concrete, James?}\kjx{no} to the Grace specification.

\ugh{One exception is that it still supports the \code{nil} value of
  Newspeak.}\kjx{who cares}

\sm{we could say many more things here, not sure we need to though, depends a bit on what we need/rely on in late text}
\kjx{so we fix it later}

\kjx{do we need to talk any more about SOMns here?}

% While not everyone has time to create their own fully-fledged virtual, recent efforts such as PyPy \rrnote{CITE Tratt} or Truffle and Graal \cite{Wurthinger2013} are brining the power of VMs to researchers and hobbyists.

% As shown by Marr et al. \cite{Marr2016}, the peak performance of benchmarks is comparable to monolithic VM's like V8 and Crystal.

\subsection{Gradual Typing}

\sm{not sure this belongs here. it depends on how the paper is framed,
but my impression was that we don't want to say it is gradual, so,
this belongs perhaps better into the related work section for discussion
after the results}
\kjx{don't think it does}

Coined by Siek and Taha \cite{Siek2006}, later added the gradual ``guarantee'' \cite{Siek2015}.

\paragraph{Languages}

\sm{not sure what is supposed to be discussed here.
Seems to detailed, seems to to be long into the beginning of the paper}

\begin{itemize}
    \item Grace \cite{Boyland2014}
    \item Typed Racket \cite{Takikawa2016}
    \item Dart \cite{Heinze2016, Mezzetti2016}
    \item Pycket (using some kind of PyPy-like design) \cite{Bauman2017}
    \item Grift (seems to be the name of a compiler) \cite{Kuhlenschmidt2018}
    \item TypeScript on HiggsVM \cite{Richards2017}
    \item Reticulated Python \cite{Vitousek2017, Greenman2017}
\end{itemize}

\paragraph{Different types of type systems}

\sm{Also rather for related work, I think}

\begin{itemize}
    \item Nominal \cite{Muehlboeck2017}
    \item Structural \cite{Richards2017} (and Grace, \cite{Boyland2014})
    \item Tag (RPython) \cite{Greenman2017}
\end{itemize}

\paragraph{Other papers}

\sm{no idea what this is about}
\kjx{me neither}

\begin{itemize}
    \item \cite{Bloom2009}
    \item \cite{Castagna2017}
    \item \cite{Stulova2016}
\end{itemize}
