%!TEX root = ../latex/paper.tex

\section{Conclusion}
\label{sec:conclusion}

%% Context
With the wide-spread use of dynamically typed languages,
research on efficient approaches to add type to systems using such languages
becomes more important.
%
%% goals
%% approach to support types in a dynamic language
%% - without significant overhead
%% - simple to implement
This work proposes design for shallow structural approach to types
that are checked at run time.
The main goal of the design was to enable an implementation that
does not introduce significant run time overhead.
A secondary goal was that it should be able to add this approach to
existing language implementations with only a small amount of work.


%% proposed technique
%% - shallow structural types
The proposed approach to shallow types considers only the names of type members.
Thus, it does not consider types on these members.
This design enables a flexible use of structural types without requiring
type parameters.
Furthermore, it enables precise feedback on type violations
without requiring blame tracking,
as it is usually needed for approaches to gradual typing
at the cost of run-time performance.

%% - type checks, optimization based on object shapes
In our design, types are checked on access to variables,
when assigning to method parameters as well as for return values.
The information on types is encoded as part of an object's shape,
which means that they already performed shape checks in a normal dynamic
language implementation can be used to check types, too.

%% results
%% + context (baseline)
We evaluated this approach with Moth, an implementation of the Grace language
on top of Truffle and Graal.
Using the \AWFY benchmarks as well as a collection of benchmarks from the
gradual typing literature, we find that our approach to dynamic type checking
introduces an overhead of 
\OverheadTypingGMeanP (min. \OverheadTypingMinP, max. \OverheadTypingMaxP)
on peak performance.
Since Moth reaches the performance of highly optimized JavaScript VM such as V8,
we believe that these results are a good indication for the low overhead of this
approach.

%% relevance
%%   where to go from here?

%% future work
In specific cases, the overhead is still significant and requires further
research to be practical. Thus, future research should investigate how the
number of dynamic type checks can be reduced without causing
the type feedback to be come too imprecise to be useful.
One approach might increase the necessary changes to a language implementation,
but avoid checking each read.
Another approach might leverage Truffle's approach to self-specialization
to propagate type requirements and avoid unnecessary checks.
%%  - investigate optimizations
%%  - might enable a full structural type system, \ie, not shallow
%%  - verify results of Richards2017 in context of highly optimizing VM
Full structural types should be further investigated, too.
We believe it should be verified that \citet{Richards2017} result generalize
to highly optimizing virtual machines,
or other optimizations for precise blame would need to be investigated.

